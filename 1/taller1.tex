\title{Termodin\'amica - M\'odulo de teor\'ia\\Taller 1}
\author{
Gabriel Sandoval \\ gfsandovalv@unal.edu.co
   \and
  Samuel Pozada\\ sgpozadab@unal.edu.co
}
\date{Febrero de 2017}
\documentclass[12pt]{article}
\usepackage{float}
\usepackage{graphicx}
\usepackage{amsmath}
\usepackage{ragged2e}
\usepackage{mathtools}
\usepackage[figurename=]{caption}  
\usepackage{subcaption}
%\usepackage[section]{~/textmf/placeins.sty}
\usepackage[a4paper,bindingoffset=0.2in,left=0.7in,right=0.7in,top=0.7in,bottom=0.7in,footskip=.25in]{geometry}
\usepackage[utf8]{inputenc}  
\renewcommand{\thesubsection}{\thesection.\alph{subsection}}
\begin{document}
\maketitle

\section{Ley cero}
Cuando los sistemas A y C están en equilibrio, se cumple la relación
\begin{equation}
4\pi nRC_{c}H-MPV=0  
\end{equation}
Cuando B y C están en equilibrio se tiene,
\begin{equation}
nR\theta M'+4\pi nRC'_cH'-M'PV=0
\end{equation}
\subsection{Temperatura empírica}
\begin{center}
  $T_A(H,M)=\dfrac{4\pi nRC_cH}{M}$; \\ 
  $T_B(H',M')=nR(\theta+\dfrac{4\pi C'_cH'}{M'})$;  \\
  $T_C(P,V)=PV$
\end{center}
\paragraph{}
Donde $n$, $R$, $C_c$, $C'_c $ y $\theta$ son constantes.
\subsection{Isotermas}

\begin{figure}[H]
  \centering
  \begin{subfigure}[b]{0.3\textwidth}
    \scalebox{.4}{% GNUPLOT: LaTeX picture
\setlength{\unitlength}{0.240900pt}
\ifx\plotpoint\undefined\newsavebox{\plotpoint}\fi
\sbox{\plotpoint}{\rule[-0.200pt]{0.400pt}{0.400pt}}%
\begin{picture}(1500,900)(0,0)
\sbox{\plotpoint}{\rule[-0.200pt]{0.400pt}{0.400pt}}%
\put(111.0,131.0){\rule[-0.200pt]{4.818pt}{0.400pt}}
\put(91,131){\makebox(0,0)[r]{$0$}}
\put(1419.0,131.0){\rule[-0.200pt]{4.818pt}{0.400pt}}
\put(111.0,277.0){\rule[-0.200pt]{4.818pt}{0.400pt}}
\put(91,277){\makebox(0,0)[r]{$1$}}
\put(1419.0,277.0){\rule[-0.200pt]{4.818pt}{0.400pt}}
\put(111.0,422.0){\rule[-0.200pt]{4.818pt}{0.400pt}}
\put(91,422){\makebox(0,0)[r]{$2$}}
\put(1419.0,422.0){\rule[-0.200pt]{4.818pt}{0.400pt}}
\put(111.0,568.0){\rule[-0.200pt]{4.818pt}{0.400pt}}
\put(91,568){\makebox(0,0)[r]{$3$}}
\put(1419.0,568.0){\rule[-0.200pt]{4.818pt}{0.400pt}}
\put(111.0,713.0){\rule[-0.200pt]{4.818pt}{0.400pt}}
\put(91,713){\makebox(0,0)[r]{$4$}}
\put(1419.0,713.0){\rule[-0.200pt]{4.818pt}{0.400pt}}
\put(111.0,859.0){\rule[-0.200pt]{4.818pt}{0.400pt}}
\put(91,859){\makebox(0,0)[r]{$5$}}
\put(1419.0,859.0){\rule[-0.200pt]{4.818pt}{0.400pt}}
\put(111.0,131.0){\rule[-0.200pt]{0.400pt}{4.818pt}}
\put(111,90){\makebox(0,0){$0$}}
\put(111.0,839.0){\rule[-0.200pt]{0.400pt}{4.818pt}}
\put(377.0,131.0){\rule[-0.200pt]{0.400pt}{4.818pt}}
\put(377,90){\makebox(0,0){$2$}}
\put(377.0,839.0){\rule[-0.200pt]{0.400pt}{4.818pt}}
\put(642.0,131.0){\rule[-0.200pt]{0.400pt}{4.818pt}}
\put(642,90){\makebox(0,0){$4$}}
\put(642.0,839.0){\rule[-0.200pt]{0.400pt}{4.818pt}}
\put(908.0,131.0){\rule[-0.200pt]{0.400pt}{4.818pt}}
\put(908,90){\makebox(0,0){$6$}}
\put(908.0,839.0){\rule[-0.200pt]{0.400pt}{4.818pt}}
\put(1173.0,131.0){\rule[-0.200pt]{0.400pt}{4.818pt}}
\put(1173,90){\makebox(0,0){$8$}}
\put(1173.0,839.0){\rule[-0.200pt]{0.400pt}{4.818pt}}
\put(1439.0,131.0){\rule[-0.200pt]{0.400pt}{4.818pt}}
\put(1439,90){\makebox(0,0){$10$}}
\put(1439.0,839.0){\rule[-0.200pt]{0.400pt}{4.818pt}}
\put(111.0,131.0){\rule[-0.200pt]{0.400pt}{175.375pt}}
\put(111.0,131.0){\rule[-0.200pt]{319.915pt}{0.400pt}}
\put(1439.0,131.0){\rule[-0.200pt]{0.400pt}{175.375pt}}
\put(111.0,859.0){\rule[-0.200pt]{319.915pt}{0.400pt}}
\put(377,713){\makebox(0,0)[l]{$M=4\pi nRC_cH/T_a$}}
\put(30,495){\makebox(0,0){$M$}}
\put(775,29){\makebox(0,0){$H$}}
\put(111,131){\usebox{\plotpoint}}
\multiput(111.00,131.59)(0.728,0.489){15}{\rule{0.678pt}{0.118pt}}
\multiput(111.00,130.17)(11.593,9.000){2}{\rule{0.339pt}{0.400pt}}
\multiput(124.00,140.59)(0.786,0.489){15}{\rule{0.722pt}{0.118pt}}
\multiput(124.00,139.17)(12.501,9.000){2}{\rule{0.361pt}{0.400pt}}
\multiput(138.00,149.58)(0.652,0.491){17}{\rule{0.620pt}{0.118pt}}
\multiput(138.00,148.17)(11.713,10.000){2}{\rule{0.310pt}{0.400pt}}
\multiput(151.00,159.59)(0.786,0.489){15}{\rule{0.722pt}{0.118pt}}
\multiput(151.00,158.17)(12.501,9.000){2}{\rule{0.361pt}{0.400pt}}
\multiput(165.00,168.59)(0.728,0.489){15}{\rule{0.678pt}{0.118pt}}
\multiput(165.00,167.17)(11.593,9.000){2}{\rule{0.339pt}{0.400pt}}
\multiput(178.00,177.59)(0.728,0.489){15}{\rule{0.678pt}{0.118pt}}
\multiput(178.00,176.17)(11.593,9.000){2}{\rule{0.339pt}{0.400pt}}
\multiput(191.00,186.58)(0.704,0.491){17}{\rule{0.660pt}{0.118pt}}
\multiput(191.00,185.17)(12.630,10.000){2}{\rule{0.330pt}{0.400pt}}
\multiput(205.00,196.59)(0.728,0.489){15}{\rule{0.678pt}{0.118pt}}
\multiput(205.00,195.17)(11.593,9.000){2}{\rule{0.339pt}{0.400pt}}
\multiput(218.00,205.59)(0.786,0.489){15}{\rule{0.722pt}{0.118pt}}
\multiput(218.00,204.17)(12.501,9.000){2}{\rule{0.361pt}{0.400pt}}
\multiput(232.00,214.59)(0.728,0.489){15}{\rule{0.678pt}{0.118pt}}
\multiput(232.00,213.17)(11.593,9.000){2}{\rule{0.339pt}{0.400pt}}
\multiput(245.00,223.58)(0.704,0.491){17}{\rule{0.660pt}{0.118pt}}
\multiput(245.00,222.17)(12.630,10.000){2}{\rule{0.330pt}{0.400pt}}
\multiput(259.00,233.59)(0.728,0.489){15}{\rule{0.678pt}{0.118pt}}
\multiput(259.00,232.17)(11.593,9.000){2}{\rule{0.339pt}{0.400pt}}
\multiput(272.00,242.59)(0.728,0.489){15}{\rule{0.678pt}{0.118pt}}
\multiput(272.00,241.17)(11.593,9.000){2}{\rule{0.339pt}{0.400pt}}
\multiput(285.00,251.59)(0.786,0.489){15}{\rule{0.722pt}{0.118pt}}
\multiput(285.00,250.17)(12.501,9.000){2}{\rule{0.361pt}{0.400pt}}
\multiput(299.00,260.58)(0.652,0.491){17}{\rule{0.620pt}{0.118pt}}
\multiput(299.00,259.17)(11.713,10.000){2}{\rule{0.310pt}{0.400pt}}
\multiput(312.00,270.59)(0.786,0.489){15}{\rule{0.722pt}{0.118pt}}
\multiput(312.00,269.17)(12.501,9.000){2}{\rule{0.361pt}{0.400pt}}
\multiput(326.00,279.59)(0.728,0.489){15}{\rule{0.678pt}{0.118pt}}
\multiput(326.00,278.17)(11.593,9.000){2}{\rule{0.339pt}{0.400pt}}
\multiput(339.00,288.59)(0.728,0.489){15}{\rule{0.678pt}{0.118pt}}
\multiput(339.00,287.17)(11.593,9.000){2}{\rule{0.339pt}{0.400pt}}
\multiput(352.00,297.58)(0.704,0.491){17}{\rule{0.660pt}{0.118pt}}
\multiput(352.00,296.17)(12.630,10.000){2}{\rule{0.330pt}{0.400pt}}
\multiput(366.00,307.59)(0.728,0.489){15}{\rule{0.678pt}{0.118pt}}
\multiput(366.00,306.17)(11.593,9.000){2}{\rule{0.339pt}{0.400pt}}
\multiput(379.00,316.59)(0.786,0.489){15}{\rule{0.722pt}{0.118pt}}
\multiput(379.00,315.17)(12.501,9.000){2}{\rule{0.361pt}{0.400pt}}
\multiput(393.00,325.59)(0.728,0.489){15}{\rule{0.678pt}{0.118pt}}
\multiput(393.00,324.17)(11.593,9.000){2}{\rule{0.339pt}{0.400pt}}
\multiput(406.00,334.58)(0.704,0.491){17}{\rule{0.660pt}{0.118pt}}
\multiput(406.00,333.17)(12.630,10.000){2}{\rule{0.330pt}{0.400pt}}
\multiput(420.00,344.59)(0.728,0.489){15}{\rule{0.678pt}{0.118pt}}
\multiput(420.00,343.17)(11.593,9.000){2}{\rule{0.339pt}{0.400pt}}
\multiput(433.00,353.59)(0.728,0.489){15}{\rule{0.678pt}{0.118pt}}
\multiput(433.00,352.17)(11.593,9.000){2}{\rule{0.339pt}{0.400pt}}
\multiput(446.00,362.59)(0.786,0.489){15}{\rule{0.722pt}{0.118pt}}
\multiput(446.00,361.17)(12.501,9.000){2}{\rule{0.361pt}{0.400pt}}
\multiput(460.00,371.59)(0.728,0.489){15}{\rule{0.678pt}{0.118pt}}
\multiput(460.00,370.17)(11.593,9.000){2}{\rule{0.339pt}{0.400pt}}
\multiput(473.00,380.58)(0.704,0.491){17}{\rule{0.660pt}{0.118pt}}
\multiput(473.00,379.17)(12.630,10.000){2}{\rule{0.330pt}{0.400pt}}
\multiput(487.00,390.59)(0.728,0.489){15}{\rule{0.678pt}{0.118pt}}
\multiput(487.00,389.17)(11.593,9.000){2}{\rule{0.339pt}{0.400pt}}
\multiput(500.00,399.59)(0.728,0.489){15}{\rule{0.678pt}{0.118pt}}
\multiput(500.00,398.17)(11.593,9.000){2}{\rule{0.339pt}{0.400pt}}
\multiput(513.00,408.59)(0.786,0.489){15}{\rule{0.722pt}{0.118pt}}
\multiput(513.00,407.17)(12.501,9.000){2}{\rule{0.361pt}{0.400pt}}
\multiput(527.00,417.58)(0.652,0.491){17}{\rule{0.620pt}{0.118pt}}
\multiput(527.00,416.17)(11.713,10.000){2}{\rule{0.310pt}{0.400pt}}
\multiput(540.00,427.59)(0.786,0.489){15}{\rule{0.722pt}{0.118pt}}
\multiput(540.00,426.17)(12.501,9.000){2}{\rule{0.361pt}{0.400pt}}
\multiput(554.00,436.59)(0.728,0.489){15}{\rule{0.678pt}{0.118pt}}
\multiput(554.00,435.17)(11.593,9.000){2}{\rule{0.339pt}{0.400pt}}
\multiput(567.00,445.59)(0.728,0.489){15}{\rule{0.678pt}{0.118pt}}
\multiput(567.00,444.17)(11.593,9.000){2}{\rule{0.339pt}{0.400pt}}
\multiput(580.00,454.58)(0.704,0.491){17}{\rule{0.660pt}{0.118pt}}
\multiput(580.00,453.17)(12.630,10.000){2}{\rule{0.330pt}{0.400pt}}
\multiput(594.00,464.59)(0.728,0.489){15}{\rule{0.678pt}{0.118pt}}
\multiput(594.00,463.17)(11.593,9.000){2}{\rule{0.339pt}{0.400pt}}
\multiput(607.00,473.59)(0.786,0.489){15}{\rule{0.722pt}{0.118pt}}
\multiput(607.00,472.17)(12.501,9.000){2}{\rule{0.361pt}{0.400pt}}
\multiput(621.00,482.59)(0.728,0.489){15}{\rule{0.678pt}{0.118pt}}
\multiput(621.00,481.17)(11.593,9.000){2}{\rule{0.339pt}{0.400pt}}
\multiput(634.00,491.58)(0.704,0.491){17}{\rule{0.660pt}{0.118pt}}
\multiput(634.00,490.17)(12.630,10.000){2}{\rule{0.330pt}{0.400pt}}
\multiput(648.00,501.59)(0.728,0.489){15}{\rule{0.678pt}{0.118pt}}
\multiput(648.00,500.17)(11.593,9.000){2}{\rule{0.339pt}{0.400pt}}
\multiput(661.00,510.59)(0.728,0.489){15}{\rule{0.678pt}{0.118pt}}
\multiput(661.00,509.17)(11.593,9.000){2}{\rule{0.339pt}{0.400pt}}
\multiput(674.00,519.59)(0.786,0.489){15}{\rule{0.722pt}{0.118pt}}
\multiput(674.00,518.17)(12.501,9.000){2}{\rule{0.361pt}{0.400pt}}
\multiput(688.00,528.58)(0.652,0.491){17}{\rule{0.620pt}{0.118pt}}
\multiput(688.00,527.17)(11.713,10.000){2}{\rule{0.310pt}{0.400pt}}
\multiput(701.00,538.59)(0.786,0.489){15}{\rule{0.722pt}{0.118pt}}
\multiput(701.00,537.17)(12.501,9.000){2}{\rule{0.361pt}{0.400pt}}
\multiput(715.00,547.59)(0.728,0.489){15}{\rule{0.678pt}{0.118pt}}
\multiput(715.00,546.17)(11.593,9.000){2}{\rule{0.339pt}{0.400pt}}
\multiput(728.00,556.59)(0.728,0.489){15}{\rule{0.678pt}{0.118pt}}
\multiput(728.00,555.17)(11.593,9.000){2}{\rule{0.339pt}{0.400pt}}
\multiput(741.00,565.58)(0.704,0.491){17}{\rule{0.660pt}{0.118pt}}
\multiput(741.00,564.17)(12.630,10.000){2}{\rule{0.330pt}{0.400pt}}
\multiput(755.00,575.59)(0.728,0.489){15}{\rule{0.678pt}{0.118pt}}
\multiput(755.00,574.17)(11.593,9.000){2}{\rule{0.339pt}{0.400pt}}
\multiput(768.00,584.59)(0.786,0.489){15}{\rule{0.722pt}{0.118pt}}
\multiput(768.00,583.17)(12.501,9.000){2}{\rule{0.361pt}{0.400pt}}
\multiput(782.00,593.59)(0.728,0.489){15}{\rule{0.678pt}{0.118pt}}
\multiput(782.00,592.17)(11.593,9.000){2}{\rule{0.339pt}{0.400pt}}
\multiput(795.00,602.58)(0.704,0.491){17}{\rule{0.660pt}{0.118pt}}
\multiput(795.00,601.17)(12.630,10.000){2}{\rule{0.330pt}{0.400pt}}
\multiput(809.00,612.59)(0.728,0.489){15}{\rule{0.678pt}{0.118pt}}
\multiput(809.00,611.17)(11.593,9.000){2}{\rule{0.339pt}{0.400pt}}
\multiput(822.00,621.59)(0.728,0.489){15}{\rule{0.678pt}{0.118pt}}
\multiput(822.00,620.17)(11.593,9.000){2}{\rule{0.339pt}{0.400pt}}
\multiput(835.00,630.59)(0.786,0.489){15}{\rule{0.722pt}{0.118pt}}
\multiput(835.00,629.17)(12.501,9.000){2}{\rule{0.361pt}{0.400pt}}
\multiput(849.00,639.59)(0.728,0.489){15}{\rule{0.678pt}{0.118pt}}
\multiput(849.00,638.17)(11.593,9.000){2}{\rule{0.339pt}{0.400pt}}
\multiput(862.00,648.58)(0.704,0.491){17}{\rule{0.660pt}{0.118pt}}
\multiput(862.00,647.17)(12.630,10.000){2}{\rule{0.330pt}{0.400pt}}
\multiput(876.00,658.59)(0.728,0.489){15}{\rule{0.678pt}{0.118pt}}
\multiput(876.00,657.17)(11.593,9.000){2}{\rule{0.339pt}{0.400pt}}
\multiput(889.00,667.59)(0.728,0.489){15}{\rule{0.678pt}{0.118pt}}
\multiput(889.00,666.17)(11.593,9.000){2}{\rule{0.339pt}{0.400pt}}
\multiput(902.00,676.59)(0.786,0.489){15}{\rule{0.722pt}{0.118pt}}
\multiput(902.00,675.17)(12.501,9.000){2}{\rule{0.361pt}{0.400pt}}
\multiput(916.00,685.58)(0.652,0.491){17}{\rule{0.620pt}{0.118pt}}
\multiput(916.00,684.17)(11.713,10.000){2}{\rule{0.310pt}{0.400pt}}
\multiput(929.00,695.59)(0.786,0.489){15}{\rule{0.722pt}{0.118pt}}
\multiput(929.00,694.17)(12.501,9.000){2}{\rule{0.361pt}{0.400pt}}
\multiput(943.00,704.59)(0.728,0.489){15}{\rule{0.678pt}{0.118pt}}
\multiput(943.00,703.17)(11.593,9.000){2}{\rule{0.339pt}{0.400pt}}
\multiput(956.00,713.59)(0.786,0.489){15}{\rule{0.722pt}{0.118pt}}
\multiput(956.00,712.17)(12.501,9.000){2}{\rule{0.361pt}{0.400pt}}
\multiput(970.00,722.58)(0.652,0.491){17}{\rule{0.620pt}{0.118pt}}
\multiput(970.00,721.17)(11.713,10.000){2}{\rule{0.310pt}{0.400pt}}
\multiput(983.00,732.59)(0.728,0.489){15}{\rule{0.678pt}{0.118pt}}
\multiput(983.00,731.17)(11.593,9.000){2}{\rule{0.339pt}{0.400pt}}
\multiput(996.00,741.59)(0.786,0.489){15}{\rule{0.722pt}{0.118pt}}
\multiput(996.00,740.17)(12.501,9.000){2}{\rule{0.361pt}{0.400pt}}
\multiput(1010.00,750.59)(0.728,0.489){15}{\rule{0.678pt}{0.118pt}}
\multiput(1010.00,749.17)(11.593,9.000){2}{\rule{0.339pt}{0.400pt}}
\multiput(1023.00,759.58)(0.704,0.491){17}{\rule{0.660pt}{0.118pt}}
\multiput(1023.00,758.17)(12.630,10.000){2}{\rule{0.330pt}{0.400pt}}
\multiput(1037.00,769.59)(0.728,0.489){15}{\rule{0.678pt}{0.118pt}}
\multiput(1037.00,768.17)(11.593,9.000){2}{\rule{0.339pt}{0.400pt}}
\multiput(1050.00,778.59)(0.728,0.489){15}{\rule{0.678pt}{0.118pt}}
\multiput(1050.00,777.17)(11.593,9.000){2}{\rule{0.339pt}{0.400pt}}
\multiput(1063.00,787.59)(0.786,0.489){15}{\rule{0.722pt}{0.118pt}}
\multiput(1063.00,786.17)(12.501,9.000){2}{\rule{0.361pt}{0.400pt}}
\multiput(1077.00,796.58)(0.652,0.491){17}{\rule{0.620pt}{0.118pt}}
\multiput(1077.00,795.17)(11.713,10.000){2}{\rule{0.310pt}{0.400pt}}
\multiput(1090.00,806.59)(0.786,0.489){15}{\rule{0.722pt}{0.118pt}}
\multiput(1090.00,805.17)(12.501,9.000){2}{\rule{0.361pt}{0.400pt}}
\multiput(1104.00,815.59)(0.728,0.489){15}{\rule{0.678pt}{0.118pt}}
\multiput(1104.00,814.17)(11.593,9.000){2}{\rule{0.339pt}{0.400pt}}
\multiput(1117.00,824.59)(0.728,0.489){15}{\rule{0.678pt}{0.118pt}}
\multiput(1117.00,823.17)(11.593,9.000){2}{\rule{0.339pt}{0.400pt}}
\multiput(1130.00,833.58)(0.704,0.491){17}{\rule{0.660pt}{0.118pt}}
\multiput(1130.00,832.17)(12.630,10.000){2}{\rule{0.330pt}{0.400pt}}
\multiput(1144.00,843.59)(0.728,0.489){15}{\rule{0.678pt}{0.118pt}}
\multiput(1144.00,842.17)(11.593,9.000){2}{\rule{0.339pt}{0.400pt}}
\multiput(1157.00,852.59)(0.798,0.485){11}{\rule{0.729pt}{0.117pt}}
\multiput(1157.00,851.17)(9.488,7.000){2}{\rule{0.364pt}{0.400pt}}
\put(111,131){\usebox{\plotpoint}}
\put(111.00,131.00){\usebox{\plotpoint}}
\put(129.92,139.54){\usebox{\plotpoint}}
\put(148.86,148.01){\usebox{\plotpoint}}
\put(167.49,157.15){\usebox{\plotpoint}}
\put(186.34,165.85){\usebox{\plotpoint}}
\put(205.35,174.16){\usebox{\plotpoint}}
\put(224.28,182.69){\usebox{\plotpoint}}
\put(242.88,191.86){\usebox{\plotpoint}}
\put(261.83,200.30){\usebox{\plotpoint}}
\put(280.67,209.00){\usebox{\plotpoint}}
\put(299.69,217.32){\usebox{\plotpoint}}
\put(318.43,226.22){\usebox{\plotpoint}}
\put(337.16,235.15){\usebox{\plotpoint}}
\put(356.06,243.74){\usebox{\plotpoint}}
\put(375.02,252.17){\usebox{\plotpoint}}
\put(394.01,260.54){\usebox{\plotpoint}}
\put(412.56,269.81){\usebox{\plotpoint}}
\put(431.50,278.31){\usebox{\plotpoint}}
\put(450.39,286.88){\usebox{\plotpoint}}
\put(469.36,295.32){\usebox{\plotpoint}}
\put(488.33,303.72){\usebox{\plotpoint}}
\put(506.81,313.14){\usebox{\plotpoint}}
\put(525.81,321.49){\usebox{\plotpoint}}
\put(544.73,330.03){\usebox{\plotpoint}}
\put(563.69,338.47){\usebox{\plotpoint}}
\put(582.15,347.92){\usebox{\plotpoint}}
\put(601.14,356.30){\usebox{\plotpoint}}
\put(620.15,364.63){\usebox{\plotpoint}}
\put(639.06,373.17){\usebox{\plotpoint}}
\put(657.71,382.23){\usebox{\plotpoint}}
\put(676.49,391.07){\usebox{\plotpoint}}
\put(695.47,399.45){\usebox{\plotpoint}}
\put(714.48,407.78){\usebox{\plotpoint}}
\put(733.17,416.78){\usebox{\plotpoint}}
\put(751.91,425.67){\usebox{\plotpoint}}
\put(770.82,434.21){\usebox{\plotpoint}}
\put(789.80,442.60){\usebox{\plotpoint}}
\put(808.82,450.92){\usebox{\plotpoint}}
\put(827.26,460.43){\usebox{\plotpoint}}
\put(846.24,468.82){\usebox{\plotpoint}}
\put(865.16,477.35){\usebox{\plotpoint}}
\put(884.13,485.75){\usebox{\plotpoint}}
\put(902.97,494.48){\usebox{\plotpoint}}
\put(921.61,503.59){\usebox{\plotpoint}}
\put(940.60,511.97){\usebox{\plotpoint}}
\put(959.52,520.51){\usebox{\plotpoint}}
\put(978.49,528.92){\usebox{\plotpoint}}
\put(996.94,538.40){\usebox{\plotpoint}}
\put(1015.94,546.74){\usebox{\plotpoint}}
\put(1034.93,555.11){\usebox{\plotpoint}}
\put(1053.81,563.76){\usebox{\plotpoint}}
\put(1072.51,572.75){\usebox{\plotpoint}}
\put(1091.30,581.56){\usebox{\plotpoint}}
\put(1110.30,589.91){\usebox{\plotpoint}}
\put(1129.14,598.61){\usebox{\plotpoint}}
\put(1148.03,607.17){\usebox{\plotpoint}}
\put(1166.72,616.16){\usebox{\plotpoint}}
\put(1185.63,624.70){\usebox{\plotpoint}}
\put(1204.63,633.06){\usebox{\plotpoint}}
\put(1223.48,641.76){\usebox{\plotpoint}}
\put(1242.35,650.34){\usebox{\plotpoint}}
\put(1261.05,659.31){\usebox{\plotpoint}}
\put(1279.95,667.90){\usebox{\plotpoint}}
\put(1298.89,676.38){\usebox{\plotpoint}}
\put(1317.81,684.91){\usebox{\plotpoint}}
\put(1336.44,694.05){\usebox{\plotpoint}}
\put(1355.41,702.46){\usebox{\plotpoint}}
\put(1374.30,711.06){\usebox{\plotpoint}}
\put(1393.25,719.53){\usebox{\plotpoint}}
\put(1411.76,728.87){\usebox{\plotpoint}}
\put(1430.77,737.20){\usebox{\plotpoint}}
\put(1439,741){\usebox{\plotpoint}}
\sbox{\plotpoint}{\rule[-0.400pt]{0.800pt}{0.800pt}}%
\put(111,131){\usebox{\plotpoint}}
\multiput(111.00,132.38)(1.768,0.560){3}{\rule{2.280pt}{0.135pt}}
\multiput(111.00,129.34)(8.268,5.000){2}{\rule{1.140pt}{0.800pt}}
\put(124,136.34){\rule{3.000pt}{0.800pt}}
\multiput(124.00,134.34)(7.773,4.000){2}{\rule{1.500pt}{0.800pt}}
\multiput(138.00,141.38)(1.768,0.560){3}{\rule{2.280pt}{0.135pt}}
\multiput(138.00,138.34)(8.268,5.000){2}{\rule{1.140pt}{0.800pt}}
\put(151,145.34){\rule{3.000pt}{0.800pt}}
\multiput(151.00,143.34)(7.773,4.000){2}{\rule{1.500pt}{0.800pt}}
\multiput(165.00,150.38)(1.768,0.560){3}{\rule{2.280pt}{0.135pt}}
\multiput(165.00,147.34)(8.268,5.000){2}{\rule{1.140pt}{0.800pt}}
\multiput(178.00,155.38)(1.768,0.560){3}{\rule{2.280pt}{0.135pt}}
\multiput(178.00,152.34)(8.268,5.000){2}{\rule{1.140pt}{0.800pt}}
\put(191,159.34){\rule{3.000pt}{0.800pt}}
\multiput(191.00,157.34)(7.773,4.000){2}{\rule{1.500pt}{0.800pt}}
\multiput(205.00,164.38)(1.768,0.560){3}{\rule{2.280pt}{0.135pt}}
\multiput(205.00,161.34)(8.268,5.000){2}{\rule{1.140pt}{0.800pt}}
\multiput(218.00,169.38)(1.936,0.560){3}{\rule{2.440pt}{0.135pt}}
\multiput(218.00,166.34)(8.936,5.000){2}{\rule{1.220pt}{0.800pt}}
\put(232,173.34){\rule{2.800pt}{0.800pt}}
\multiput(232.00,171.34)(7.188,4.000){2}{\rule{1.400pt}{0.800pt}}
\multiput(245.00,178.38)(1.936,0.560){3}{\rule{2.440pt}{0.135pt}}
\multiput(245.00,175.34)(8.936,5.000){2}{\rule{1.220pt}{0.800pt}}
\put(259,182.34){\rule{2.800pt}{0.800pt}}
\multiput(259.00,180.34)(7.188,4.000){2}{\rule{1.400pt}{0.800pt}}
\multiput(272.00,187.38)(1.768,0.560){3}{\rule{2.280pt}{0.135pt}}
\multiput(272.00,184.34)(8.268,5.000){2}{\rule{1.140pt}{0.800pt}}
\multiput(285.00,192.38)(1.936,0.560){3}{\rule{2.440pt}{0.135pt}}
\multiput(285.00,189.34)(8.936,5.000){2}{\rule{1.220pt}{0.800pt}}
\put(299,196.34){\rule{2.800pt}{0.800pt}}
\multiput(299.00,194.34)(7.188,4.000){2}{\rule{1.400pt}{0.800pt}}
\multiput(312.00,201.38)(1.936,0.560){3}{\rule{2.440pt}{0.135pt}}
\multiput(312.00,198.34)(8.936,5.000){2}{\rule{1.220pt}{0.800pt}}
\multiput(326.00,206.38)(1.768,0.560){3}{\rule{2.280pt}{0.135pt}}
\multiput(326.00,203.34)(8.268,5.000){2}{\rule{1.140pt}{0.800pt}}
\put(339,210.34){\rule{2.800pt}{0.800pt}}
\multiput(339.00,208.34)(7.188,4.000){2}{\rule{1.400pt}{0.800pt}}
\multiput(352.00,215.38)(1.936,0.560){3}{\rule{2.440pt}{0.135pt}}
\multiput(352.00,212.34)(8.936,5.000){2}{\rule{1.220pt}{0.800pt}}
\put(366,219.34){\rule{2.800pt}{0.800pt}}
\multiput(366.00,217.34)(7.188,4.000){2}{\rule{1.400pt}{0.800pt}}
\multiput(379.00,224.38)(1.936,0.560){3}{\rule{2.440pt}{0.135pt}}
\multiput(379.00,221.34)(8.936,5.000){2}{\rule{1.220pt}{0.800pt}}
\multiput(393.00,229.38)(1.768,0.560){3}{\rule{2.280pt}{0.135pt}}
\multiput(393.00,226.34)(8.268,5.000){2}{\rule{1.140pt}{0.800pt}}
\put(406,233.34){\rule{3.000pt}{0.800pt}}
\multiput(406.00,231.34)(7.773,4.000){2}{\rule{1.500pt}{0.800pt}}
\multiput(420.00,238.38)(1.768,0.560){3}{\rule{2.280pt}{0.135pt}}
\multiput(420.00,235.34)(8.268,5.000){2}{\rule{1.140pt}{0.800pt}}
\multiput(433.00,243.38)(1.768,0.560){3}{\rule{2.280pt}{0.135pt}}
\multiput(433.00,240.34)(8.268,5.000){2}{\rule{1.140pt}{0.800pt}}
\put(446,247.34){\rule{3.000pt}{0.800pt}}
\multiput(446.00,245.34)(7.773,4.000){2}{\rule{1.500pt}{0.800pt}}
\multiput(460.00,252.38)(1.768,0.560){3}{\rule{2.280pt}{0.135pt}}
\multiput(460.00,249.34)(8.268,5.000){2}{\rule{1.140pt}{0.800pt}}
\put(473,256.34){\rule{3.000pt}{0.800pt}}
\multiput(473.00,254.34)(7.773,4.000){2}{\rule{1.500pt}{0.800pt}}
\multiput(487.00,261.38)(1.768,0.560){3}{\rule{2.280pt}{0.135pt}}
\multiput(487.00,258.34)(8.268,5.000){2}{\rule{1.140pt}{0.800pt}}
\multiput(500.00,266.38)(1.768,0.560){3}{\rule{2.280pt}{0.135pt}}
\multiput(500.00,263.34)(8.268,5.000){2}{\rule{1.140pt}{0.800pt}}
\put(513,270.34){\rule{3.000pt}{0.800pt}}
\multiput(513.00,268.34)(7.773,4.000){2}{\rule{1.500pt}{0.800pt}}
\multiput(527.00,275.38)(1.768,0.560){3}{\rule{2.280pt}{0.135pt}}
\multiput(527.00,272.34)(8.268,5.000){2}{\rule{1.140pt}{0.800pt}}
\put(540,279.34){\rule{3.000pt}{0.800pt}}
\multiput(540.00,277.34)(7.773,4.000){2}{\rule{1.500pt}{0.800pt}}
\multiput(554.00,284.38)(1.768,0.560){3}{\rule{2.280pt}{0.135pt}}
\multiput(554.00,281.34)(8.268,5.000){2}{\rule{1.140pt}{0.800pt}}
\multiput(567.00,289.38)(1.768,0.560){3}{\rule{2.280pt}{0.135pt}}
\multiput(567.00,286.34)(8.268,5.000){2}{\rule{1.140pt}{0.800pt}}
\put(580,293.34){\rule{3.000pt}{0.800pt}}
\multiput(580.00,291.34)(7.773,4.000){2}{\rule{1.500pt}{0.800pt}}
\multiput(594.00,298.38)(1.768,0.560){3}{\rule{2.280pt}{0.135pt}}
\multiput(594.00,295.34)(8.268,5.000){2}{\rule{1.140pt}{0.800pt}}
\multiput(607.00,303.38)(1.936,0.560){3}{\rule{2.440pt}{0.135pt}}
\multiput(607.00,300.34)(8.936,5.000){2}{\rule{1.220pt}{0.800pt}}
\put(621,307.34){\rule{2.800pt}{0.800pt}}
\multiput(621.00,305.34)(7.188,4.000){2}{\rule{1.400pt}{0.800pt}}
\multiput(634.00,312.38)(1.936,0.560){3}{\rule{2.440pt}{0.135pt}}
\multiput(634.00,309.34)(8.936,5.000){2}{\rule{1.220pt}{0.800pt}}
\put(648,316.34){\rule{2.800pt}{0.800pt}}
\multiput(648.00,314.34)(7.188,4.000){2}{\rule{1.400pt}{0.800pt}}
\multiput(661.00,321.38)(1.768,0.560){3}{\rule{2.280pt}{0.135pt}}
\multiput(661.00,318.34)(8.268,5.000){2}{\rule{1.140pt}{0.800pt}}
\multiput(674.00,326.38)(1.936,0.560){3}{\rule{2.440pt}{0.135pt}}
\multiput(674.00,323.34)(8.936,5.000){2}{\rule{1.220pt}{0.800pt}}
\put(688,330.34){\rule{2.800pt}{0.800pt}}
\multiput(688.00,328.34)(7.188,4.000){2}{\rule{1.400pt}{0.800pt}}
\multiput(701.00,335.38)(1.936,0.560){3}{\rule{2.440pt}{0.135pt}}
\multiput(701.00,332.34)(8.936,5.000){2}{\rule{1.220pt}{0.800pt}}
\multiput(715.00,340.38)(1.768,0.560){3}{\rule{2.280pt}{0.135pt}}
\multiput(715.00,337.34)(8.268,5.000){2}{\rule{1.140pt}{0.800pt}}
\put(728,344.34){\rule{2.800pt}{0.800pt}}
\multiput(728.00,342.34)(7.188,4.000){2}{\rule{1.400pt}{0.800pt}}
\multiput(741.00,349.38)(1.936,0.560){3}{\rule{2.440pt}{0.135pt}}
\multiput(741.00,346.34)(8.936,5.000){2}{\rule{1.220pt}{0.800pt}}
\put(755,353.34){\rule{2.800pt}{0.800pt}}
\multiput(755.00,351.34)(7.188,4.000){2}{\rule{1.400pt}{0.800pt}}
\multiput(768.00,358.38)(1.936,0.560){3}{\rule{2.440pt}{0.135pt}}
\multiput(768.00,355.34)(8.936,5.000){2}{\rule{1.220pt}{0.800pt}}
\multiput(782.00,363.38)(1.768,0.560){3}{\rule{2.280pt}{0.135pt}}
\multiput(782.00,360.34)(8.268,5.000){2}{\rule{1.140pt}{0.800pt}}
\put(795,367.34){\rule{3.000pt}{0.800pt}}
\multiput(795.00,365.34)(7.773,4.000){2}{\rule{1.500pt}{0.800pt}}
\multiput(809.00,372.38)(1.768,0.560){3}{\rule{2.280pt}{0.135pt}}
\multiput(809.00,369.34)(8.268,5.000){2}{\rule{1.140pt}{0.800pt}}
\put(822,376.34){\rule{2.800pt}{0.800pt}}
\multiput(822.00,374.34)(7.188,4.000){2}{\rule{1.400pt}{0.800pt}}
\multiput(835.00,381.38)(1.936,0.560){3}{\rule{2.440pt}{0.135pt}}
\multiput(835.00,378.34)(8.936,5.000){2}{\rule{1.220pt}{0.800pt}}
\multiput(849.00,386.38)(1.768,0.560){3}{\rule{2.280pt}{0.135pt}}
\multiput(849.00,383.34)(8.268,5.000){2}{\rule{1.140pt}{0.800pt}}
\put(862,390.34){\rule{3.000pt}{0.800pt}}
\multiput(862.00,388.34)(7.773,4.000){2}{\rule{1.500pt}{0.800pt}}
\multiput(876.00,395.38)(1.768,0.560){3}{\rule{2.280pt}{0.135pt}}
\multiput(876.00,392.34)(8.268,5.000){2}{\rule{1.140pt}{0.800pt}}
\multiput(889.00,400.38)(1.768,0.560){3}{\rule{2.280pt}{0.135pt}}
\multiput(889.00,397.34)(8.268,5.000){2}{\rule{1.140pt}{0.800pt}}
\put(902,404.34){\rule{3.000pt}{0.800pt}}
\multiput(902.00,402.34)(7.773,4.000){2}{\rule{1.500pt}{0.800pt}}
\multiput(916.00,409.38)(1.768,0.560){3}{\rule{2.280pt}{0.135pt}}
\multiput(916.00,406.34)(8.268,5.000){2}{\rule{1.140pt}{0.800pt}}
\put(929,413.34){\rule{3.000pt}{0.800pt}}
\multiput(929.00,411.34)(7.773,4.000){2}{\rule{1.500pt}{0.800pt}}
\multiput(943.00,418.38)(1.768,0.560){3}{\rule{2.280pt}{0.135pt}}
\multiput(943.00,415.34)(8.268,5.000){2}{\rule{1.140pt}{0.800pt}}
\multiput(956.00,423.38)(1.936,0.560){3}{\rule{2.440pt}{0.135pt}}
\multiput(956.00,420.34)(8.936,5.000){2}{\rule{1.220pt}{0.800pt}}
\put(970,427.34){\rule{2.800pt}{0.800pt}}
\multiput(970.00,425.34)(7.188,4.000){2}{\rule{1.400pt}{0.800pt}}
\multiput(983.00,432.38)(1.768,0.560){3}{\rule{2.280pt}{0.135pt}}
\multiput(983.00,429.34)(8.268,5.000){2}{\rule{1.140pt}{0.800pt}}
\multiput(996.00,437.38)(1.936,0.560){3}{\rule{2.440pt}{0.135pt}}
\multiput(996.00,434.34)(8.936,5.000){2}{\rule{1.220pt}{0.800pt}}
\put(1010,441.34){\rule{2.800pt}{0.800pt}}
\multiput(1010.00,439.34)(7.188,4.000){2}{\rule{1.400pt}{0.800pt}}
\multiput(1023.00,446.38)(1.936,0.560){3}{\rule{2.440pt}{0.135pt}}
\multiput(1023.00,443.34)(8.936,5.000){2}{\rule{1.220pt}{0.800pt}}
\put(1037,450.34){\rule{2.800pt}{0.800pt}}
\multiput(1037.00,448.34)(7.188,4.000){2}{\rule{1.400pt}{0.800pt}}
\multiput(1050.00,455.38)(1.768,0.560){3}{\rule{2.280pt}{0.135pt}}
\multiput(1050.00,452.34)(8.268,5.000){2}{\rule{1.140pt}{0.800pt}}
\multiput(1063.00,460.38)(1.936,0.560){3}{\rule{2.440pt}{0.135pt}}
\multiput(1063.00,457.34)(8.936,5.000){2}{\rule{1.220pt}{0.800pt}}
\put(1077,464.34){\rule{2.800pt}{0.800pt}}
\multiput(1077.00,462.34)(7.188,4.000){2}{\rule{1.400pt}{0.800pt}}
\multiput(1090.00,469.38)(1.936,0.560){3}{\rule{2.440pt}{0.135pt}}
\multiput(1090.00,466.34)(8.936,5.000){2}{\rule{1.220pt}{0.800pt}}
\multiput(1104.00,474.38)(1.768,0.560){3}{\rule{2.280pt}{0.135pt}}
\multiput(1104.00,471.34)(8.268,5.000){2}{\rule{1.140pt}{0.800pt}}
\put(1117,478.34){\rule{2.800pt}{0.800pt}}
\multiput(1117.00,476.34)(7.188,4.000){2}{\rule{1.400pt}{0.800pt}}
\multiput(1130.00,483.38)(1.936,0.560){3}{\rule{2.440pt}{0.135pt}}
\multiput(1130.00,480.34)(8.936,5.000){2}{\rule{1.220pt}{0.800pt}}
\put(1144,487.34){\rule{2.800pt}{0.800pt}}
\multiput(1144.00,485.34)(7.188,4.000){2}{\rule{1.400pt}{0.800pt}}
\multiput(1157.00,492.38)(1.936,0.560){3}{\rule{2.440pt}{0.135pt}}
\multiput(1157.00,489.34)(8.936,5.000){2}{\rule{1.220pt}{0.800pt}}
\multiput(1171.00,497.38)(1.768,0.560){3}{\rule{2.280pt}{0.135pt}}
\multiput(1171.00,494.34)(8.268,5.000){2}{\rule{1.140pt}{0.800pt}}
\put(1184,501.34){\rule{3.000pt}{0.800pt}}
\multiput(1184.00,499.34)(7.773,4.000){2}{\rule{1.500pt}{0.800pt}}
\multiput(1198.00,506.38)(1.768,0.560){3}{\rule{2.280pt}{0.135pt}}
\multiput(1198.00,503.34)(8.268,5.000){2}{\rule{1.140pt}{0.800pt}}
\put(1211,510.34){\rule{2.800pt}{0.800pt}}
\multiput(1211.00,508.34)(7.188,4.000){2}{\rule{1.400pt}{0.800pt}}
\multiput(1224.00,515.38)(1.936,0.560){3}{\rule{2.440pt}{0.135pt}}
\multiput(1224.00,512.34)(8.936,5.000){2}{\rule{1.220pt}{0.800pt}}
\multiput(1238.00,520.38)(1.768,0.560){3}{\rule{2.280pt}{0.135pt}}
\multiput(1238.00,517.34)(8.268,5.000){2}{\rule{1.140pt}{0.800pt}}
\put(1251,524.34){\rule{3.000pt}{0.800pt}}
\multiput(1251.00,522.34)(7.773,4.000){2}{\rule{1.500pt}{0.800pt}}
\multiput(1265.00,529.38)(1.768,0.560){3}{\rule{2.280pt}{0.135pt}}
\multiput(1265.00,526.34)(8.268,5.000){2}{\rule{1.140pt}{0.800pt}}
\multiput(1278.00,534.38)(1.768,0.560){3}{\rule{2.280pt}{0.135pt}}
\multiput(1278.00,531.34)(8.268,5.000){2}{\rule{1.140pt}{0.800pt}}
\put(1291,538.34){\rule{3.000pt}{0.800pt}}
\multiput(1291.00,536.34)(7.773,4.000){2}{\rule{1.500pt}{0.800pt}}
\multiput(1305.00,543.38)(1.768,0.560){3}{\rule{2.280pt}{0.135pt}}
\multiput(1305.00,540.34)(8.268,5.000){2}{\rule{1.140pt}{0.800pt}}
\put(1318,547.34){\rule{3.000pt}{0.800pt}}
\multiput(1318.00,545.34)(7.773,4.000){2}{\rule{1.500pt}{0.800pt}}
\multiput(1332.00,552.38)(1.768,0.560){3}{\rule{2.280pt}{0.135pt}}
\multiput(1332.00,549.34)(8.268,5.000){2}{\rule{1.140pt}{0.800pt}}
\multiput(1345.00,557.38)(1.936,0.560){3}{\rule{2.440pt}{0.135pt}}
\multiput(1345.00,554.34)(8.936,5.000){2}{\rule{1.220pt}{0.800pt}}
\put(1359,561.34){\rule{2.800pt}{0.800pt}}
\multiput(1359.00,559.34)(7.188,4.000){2}{\rule{1.400pt}{0.800pt}}
\multiput(1372.00,566.38)(1.768,0.560){3}{\rule{2.280pt}{0.135pt}}
\multiput(1372.00,563.34)(8.268,5.000){2}{\rule{1.140pt}{0.800pt}}
\multiput(1385.00,571.38)(1.936,0.560){3}{\rule{2.440pt}{0.135pt}}
\multiput(1385.00,568.34)(8.936,5.000){2}{\rule{1.220pt}{0.800pt}}
\put(1399,575.34){\rule{2.800pt}{0.800pt}}
\multiput(1399.00,573.34)(7.188,4.000){2}{\rule{1.400pt}{0.800pt}}
\multiput(1412.00,580.38)(1.936,0.560){3}{\rule{2.440pt}{0.135pt}}
\multiput(1412.00,577.34)(8.936,5.000){2}{\rule{1.220pt}{0.800pt}}
\put(1426,584.34){\rule{2.800pt}{0.800pt}}
\multiput(1426.00,582.34)(7.188,4.000){2}{\rule{1.400pt}{0.800pt}}
\sbox{\plotpoint}{\rule[-0.500pt]{1.000pt}{1.000pt}}%
\put(111,131){\usebox{\plotpoint}}
\put(111.00,131.00){\usebox{\plotpoint}}
\put(131.00,136.50){\usebox{\plotpoint}}
\put(150.99,142.00){\usebox{\plotpoint}}
\put(171.03,147.39){\usebox{\plotpoint}}
\put(191.00,153.00){\usebox{\plotpoint}}
\put(210.92,158.82){\usebox{\plotpoint}}
\put(231.05,163.80){\usebox{\plotpoint}}
\put(250.95,169.70){\usebox{\plotpoint}}
\put(271.06,174.78){\usebox{\plotpoint}}
\put(290.95,180.70){\usebox{\plotpoint}}
\put(311.07,185.79){\usebox{\plotpoint}}
\put(331.01,191.54){\usebox{\plotpoint}}
\put(350.85,197.64){\usebox{\plotpoint}}
\put(371.00,202.54){\usebox{\plotpoint}}
\put(390.91,208.40){\usebox{\plotpoint}}
\put(411.04,213.44){\usebox{\plotpoint}}
\put(430.93,219.36){\usebox{\plotpoint}}
\put(451.04,224.44){\usebox{\plotpoint}}
\put(470.93,230.36){\usebox{\plotpoint}}
\put(491.09,235.26){\usebox{\plotpoint}}
\put(510.93,241.36){\usebox{\plotpoint}}
\put(530.92,246.91){\usebox{\plotpoint}}
\put(551.00,252.14){\usebox{\plotpoint}}
\put(570.93,257.91){\usebox{\plotpoint}}
\put(591.01,263.14){\usebox{\plotpoint}}
\put(610.95,268.85){\usebox{\plotpoint}}
\put(631.01,274.08){\usebox{\plotpoint}}
\put(650.94,279.90){\usebox{\plotpoint}}
\put(670.96,285.30){\usebox{\plotpoint}}
\put(690.94,290.91){\usebox{\plotpoint}}
\put(711.01,296.14){\usebox{\plotpoint}}
\put(730.93,301.90){\usebox{\plotpoint}}
\put(751.00,307.14){\usebox{\plotpoint}}
\put(770.94,312.84){\usebox{\plotpoint}}
\put(790.85,318.72){\usebox{\plotpoint}}
\put(811.00,323.61){\usebox{\plotpoint}}
\put(830.84,329.72){\usebox{\plotpoint}}
\put(850.99,334.61){\usebox{\plotpoint}}
\put(870.88,340.54){\usebox{\plotpoint}}
\put(891.00,345.61){\usebox{\plotpoint}}
\put(910.89,351.54){\usebox{\plotpoint}}
\put(931.02,356.58){\usebox{\plotpoint}}
\put(950.93,362.44){\usebox{\plotpoint}}
\put(970.86,368.20){\usebox{\plotpoint}}
\put(990.93,373.44){\usebox{\plotpoint}}
\put(1010.87,379.20){\usebox{\plotpoint}}
\put(1030.99,384.28){\usebox{\plotpoint}}
\put(1050.88,390.20){\usebox{\plotpoint}}
\put(1071.00,395.28){\usebox{\plotpoint}}
\put(1090.87,401.25){\usebox{\plotpoint}}
\put(1110.92,406.60){\usebox{\plotpoint}}
\put(1130.88,412.25){\usebox{\plotpoint}}
\put(1150.93,417.60){\usebox{\plotpoint}}
\put(1170.97,422.99){\usebox{\plotpoint}}
\put(1190.96,428.49){\usebox{\plotpoint}}
\put(1210.96,433.99){\usebox{\plotpoint}}
\put(1230.95,439.49){\usebox{\plotpoint}}
\put(1250.95,444.98){\usebox{\plotpoint}}
\put(1270.87,450.81){\usebox{\plotpoint}}
\put(1290.95,455.99){\usebox{\plotpoint}}
\put(1310.88,461.81){\usebox{\plotpoint}}
\put(1331.01,466.79){\usebox{\plotpoint}}
\put(1350.90,472.69){\usebox{\plotpoint}}
\put(1371.02,477.77){\usebox{\plotpoint}}
\put(1390.91,483.69){\usebox{\plotpoint}}
\put(1410.80,489.63){\usebox{\plotpoint}}
\put(1430.95,494.52){\usebox{\plotpoint}}
\put(1439,497){\usebox{\plotpoint}}
\sbox{\plotpoint}{\rule[-0.200pt]{0.400pt}{0.400pt}}%
\put(111.0,131.0){\rule[-0.200pt]{0.400pt}{175.375pt}}
\put(111.0,131.0){\rule[-0.200pt]{319.915pt}{0.400pt}}
\put(1439.0,131.0){\rule[-0.200pt]{0.400pt}{175.375pt}}
\put(111.0,859.0){\rule[-0.200pt]{319.915pt}{0.400pt}}
\end{picture}
}
    \caption{Sistema A}
  \end{subfigure}
  \begin{subfigure}[b]{0.3\textwidth}
    \scalebox{.4}{% GNUPLOT: LaTeX picture
\setlength{\unitlength}{0.240900pt}
\ifx\plotpoint\undefined\newsavebox{\plotpoint}\fi
\sbox{\plotpoint}{\rule[-0.200pt]{0.400pt}{0.400pt}}%
\begin{picture}(1500,900)(0,0)
\sbox{\plotpoint}{\rule[-0.200pt]{0.400pt}{0.400pt}}%
\put(111.0,131.0){\rule[-0.200pt]{4.818pt}{0.400pt}}
\put(91,131){\makebox(0,0)[r]{$0$}}
\put(1419.0,131.0){\rule[-0.200pt]{4.818pt}{0.400pt}}
\put(111.0,277.0){\rule[-0.200pt]{4.818pt}{0.400pt}}
\put(91,277){\makebox(0,0)[r]{$1$}}
\put(1419.0,277.0){\rule[-0.200pt]{4.818pt}{0.400pt}}
\put(111.0,422.0){\rule[-0.200pt]{4.818pt}{0.400pt}}
\put(91,422){\makebox(0,0)[r]{$2$}}
\put(1419.0,422.0){\rule[-0.200pt]{4.818pt}{0.400pt}}
\put(111.0,568.0){\rule[-0.200pt]{4.818pt}{0.400pt}}
\put(91,568){\makebox(0,0)[r]{$3$}}
\put(1419.0,568.0){\rule[-0.200pt]{4.818pt}{0.400pt}}
\put(111.0,713.0){\rule[-0.200pt]{4.818pt}{0.400pt}}
\put(91,713){\makebox(0,0)[r]{$4$}}
\put(1419.0,713.0){\rule[-0.200pt]{4.818pt}{0.400pt}}
\put(111.0,859.0){\rule[-0.200pt]{4.818pt}{0.400pt}}
\put(91,859){\makebox(0,0)[r]{$5$}}
\put(1419.0,859.0){\rule[-0.200pt]{4.818pt}{0.400pt}}
\put(111.0,131.0){\rule[-0.200pt]{0.400pt}{4.818pt}}
\put(111,90){\makebox(0,0){$0$}}
\put(111.0,839.0){\rule[-0.200pt]{0.400pt}{4.818pt}}
\put(377.0,131.0){\rule[-0.200pt]{0.400pt}{4.818pt}}
\put(377,90){\makebox(0,0){$2$}}
\put(377.0,839.0){\rule[-0.200pt]{0.400pt}{4.818pt}}
\put(642.0,131.0){\rule[-0.200pt]{0.400pt}{4.818pt}}
\put(642,90){\makebox(0,0){$4$}}
\put(642.0,839.0){\rule[-0.200pt]{0.400pt}{4.818pt}}
\put(908.0,131.0){\rule[-0.200pt]{0.400pt}{4.818pt}}
\put(908,90){\makebox(0,0){$6$}}
\put(908.0,839.0){\rule[-0.200pt]{0.400pt}{4.818pt}}
\put(1173.0,131.0){\rule[-0.200pt]{0.400pt}{4.818pt}}
\put(1173,90){\makebox(0,0){$8$}}
\put(1173.0,839.0){\rule[-0.200pt]{0.400pt}{4.818pt}}
\put(1439.0,131.0){\rule[-0.200pt]{0.400pt}{4.818pt}}
\put(1439,90){\makebox(0,0){$10$}}
\put(1439.0,839.0){\rule[-0.200pt]{0.400pt}{4.818pt}}
\put(111.0,131.0){\rule[-0.200pt]{0.400pt}{175.375pt}}
\put(111.0,131.0){\rule[-0.200pt]{319.915pt}{0.400pt}}
\put(1439.0,131.0){\rule[-0.200pt]{0.400pt}{175.375pt}}
\put(111.0,859.0){\rule[-0.200pt]{319.915pt}{0.400pt}}
\put(377,713){\makebox(0,0)[l]{$M'=\dfrac{4\\piC'_cH'}{T_a-nR\theta}$}}
\put(30,495){\makebox(0,0){$M'$}}
\put(775,29){\makebox(0,0){$H'$}}
\put(111,131){\usebox{\plotpoint}}
\multiput(111.00,131.58)(0.652,0.491){17}{\rule{0.620pt}{0.118pt}}
\multiput(111.00,130.17)(11.713,10.000){2}{\rule{0.310pt}{0.400pt}}
\multiput(124.00,141.59)(0.786,0.489){15}{\rule{0.722pt}{0.118pt}}
\multiput(124.00,140.17)(12.501,9.000){2}{\rule{0.361pt}{0.400pt}}
\multiput(138.00,150.58)(0.652,0.491){17}{\rule{0.620pt}{0.118pt}}
\multiput(138.00,149.17)(11.713,10.000){2}{\rule{0.310pt}{0.400pt}}
\multiput(151.00,160.58)(0.704,0.491){17}{\rule{0.660pt}{0.118pt}}
\multiput(151.00,159.17)(12.630,10.000){2}{\rule{0.330pt}{0.400pt}}
\multiput(165.00,170.58)(0.652,0.491){17}{\rule{0.620pt}{0.118pt}}
\multiput(165.00,169.17)(11.713,10.000){2}{\rule{0.310pt}{0.400pt}}
\multiput(178.00,180.59)(0.728,0.489){15}{\rule{0.678pt}{0.118pt}}
\multiput(178.00,179.17)(11.593,9.000){2}{\rule{0.339pt}{0.400pt}}
\multiput(191.00,189.58)(0.704,0.491){17}{\rule{0.660pt}{0.118pt}}
\multiput(191.00,188.17)(12.630,10.000){2}{\rule{0.330pt}{0.400pt}}
\multiput(205.00,199.58)(0.652,0.491){17}{\rule{0.620pt}{0.118pt}}
\multiput(205.00,198.17)(11.713,10.000){2}{\rule{0.310pt}{0.400pt}}
\multiput(218.00,209.58)(0.704,0.491){17}{\rule{0.660pt}{0.118pt}}
\multiput(218.00,208.17)(12.630,10.000){2}{\rule{0.330pt}{0.400pt}}
\multiput(232.00,219.59)(0.728,0.489){15}{\rule{0.678pt}{0.118pt}}
\multiput(232.00,218.17)(11.593,9.000){2}{\rule{0.339pt}{0.400pt}}
\multiput(245.00,228.58)(0.704,0.491){17}{\rule{0.660pt}{0.118pt}}
\multiput(245.00,227.17)(12.630,10.000){2}{\rule{0.330pt}{0.400pt}}
\multiput(259.00,238.58)(0.652,0.491){17}{\rule{0.620pt}{0.118pt}}
\multiput(259.00,237.17)(11.713,10.000){2}{\rule{0.310pt}{0.400pt}}
\multiput(272.00,248.59)(0.728,0.489){15}{\rule{0.678pt}{0.118pt}}
\multiput(272.00,247.17)(11.593,9.000){2}{\rule{0.339pt}{0.400pt}}
\multiput(285.00,257.58)(0.704,0.491){17}{\rule{0.660pt}{0.118pt}}
\multiput(285.00,256.17)(12.630,10.000){2}{\rule{0.330pt}{0.400pt}}
\multiput(299.00,267.58)(0.652,0.491){17}{\rule{0.620pt}{0.118pt}}
\multiput(299.00,266.17)(11.713,10.000){2}{\rule{0.310pt}{0.400pt}}
\multiput(312.00,277.58)(0.704,0.491){17}{\rule{0.660pt}{0.118pt}}
\multiput(312.00,276.17)(12.630,10.000){2}{\rule{0.330pt}{0.400pt}}
\multiput(326.00,287.59)(0.728,0.489){15}{\rule{0.678pt}{0.118pt}}
\multiput(326.00,286.17)(11.593,9.000){2}{\rule{0.339pt}{0.400pt}}
\multiput(339.00,296.58)(0.652,0.491){17}{\rule{0.620pt}{0.118pt}}
\multiput(339.00,295.17)(11.713,10.000){2}{\rule{0.310pt}{0.400pt}}
\multiput(352.00,306.58)(0.704,0.491){17}{\rule{0.660pt}{0.118pt}}
\multiput(352.00,305.17)(12.630,10.000){2}{\rule{0.330pt}{0.400pt}}
\multiput(366.00,316.58)(0.652,0.491){17}{\rule{0.620pt}{0.118pt}}
\multiput(366.00,315.17)(11.713,10.000){2}{\rule{0.310pt}{0.400pt}}
\multiput(379.00,326.59)(0.786,0.489){15}{\rule{0.722pt}{0.118pt}}
\multiput(379.00,325.17)(12.501,9.000){2}{\rule{0.361pt}{0.400pt}}
\multiput(393.00,335.58)(0.652,0.491){17}{\rule{0.620pt}{0.118pt}}
\multiput(393.00,334.17)(11.713,10.000){2}{\rule{0.310pt}{0.400pt}}
\multiput(406.00,345.58)(0.704,0.491){17}{\rule{0.660pt}{0.118pt}}
\multiput(406.00,344.17)(12.630,10.000){2}{\rule{0.330pt}{0.400pt}}
\multiput(420.00,355.59)(0.728,0.489){15}{\rule{0.678pt}{0.118pt}}
\multiput(420.00,354.17)(11.593,9.000){2}{\rule{0.339pt}{0.400pt}}
\multiput(433.00,364.58)(0.652,0.491){17}{\rule{0.620pt}{0.118pt}}
\multiput(433.00,363.17)(11.713,10.000){2}{\rule{0.310pt}{0.400pt}}
\multiput(446.00,374.58)(0.704,0.491){17}{\rule{0.660pt}{0.118pt}}
\multiput(446.00,373.17)(12.630,10.000){2}{\rule{0.330pt}{0.400pt}}
\multiput(460.00,384.58)(0.652,0.491){17}{\rule{0.620pt}{0.118pt}}
\multiput(460.00,383.17)(11.713,10.000){2}{\rule{0.310pt}{0.400pt}}
\multiput(473.00,394.59)(0.786,0.489){15}{\rule{0.722pt}{0.118pt}}
\multiput(473.00,393.17)(12.501,9.000){2}{\rule{0.361pt}{0.400pt}}
\multiput(487.00,403.58)(0.652,0.491){17}{\rule{0.620pt}{0.118pt}}
\multiput(487.00,402.17)(11.713,10.000){2}{\rule{0.310pt}{0.400pt}}
\multiput(500.00,413.58)(0.652,0.491){17}{\rule{0.620pt}{0.118pt}}
\multiput(500.00,412.17)(11.713,10.000){2}{\rule{0.310pt}{0.400pt}}
\multiput(513.00,423.58)(0.704,0.491){17}{\rule{0.660pt}{0.118pt}}
\multiput(513.00,422.17)(12.630,10.000){2}{\rule{0.330pt}{0.400pt}}
\multiput(527.00,433.59)(0.728,0.489){15}{\rule{0.678pt}{0.118pt}}
\multiput(527.00,432.17)(11.593,9.000){2}{\rule{0.339pt}{0.400pt}}
\multiput(540.00,442.58)(0.704,0.491){17}{\rule{0.660pt}{0.118pt}}
\multiput(540.00,441.17)(12.630,10.000){2}{\rule{0.330pt}{0.400pt}}
\multiput(554.00,452.58)(0.652,0.491){17}{\rule{0.620pt}{0.118pt}}
\multiput(554.00,451.17)(11.713,10.000){2}{\rule{0.310pt}{0.400pt}}
\multiput(567.00,462.59)(0.728,0.489){15}{\rule{0.678pt}{0.118pt}}
\multiput(567.00,461.17)(11.593,9.000){2}{\rule{0.339pt}{0.400pt}}
\multiput(580.00,471.58)(0.704,0.491){17}{\rule{0.660pt}{0.118pt}}
\multiput(580.00,470.17)(12.630,10.000){2}{\rule{0.330pt}{0.400pt}}
\multiput(594.00,481.58)(0.652,0.491){17}{\rule{0.620pt}{0.118pt}}
\multiput(594.00,480.17)(11.713,10.000){2}{\rule{0.310pt}{0.400pt}}
\multiput(607.00,491.58)(0.704,0.491){17}{\rule{0.660pt}{0.118pt}}
\multiput(607.00,490.17)(12.630,10.000){2}{\rule{0.330pt}{0.400pt}}
\multiput(621.00,501.59)(0.728,0.489){15}{\rule{0.678pt}{0.118pt}}
\multiput(621.00,500.17)(11.593,9.000){2}{\rule{0.339pt}{0.400pt}}
\multiput(634.00,510.58)(0.704,0.491){17}{\rule{0.660pt}{0.118pt}}
\multiput(634.00,509.17)(12.630,10.000){2}{\rule{0.330pt}{0.400pt}}
\multiput(648.00,520.58)(0.652,0.491){17}{\rule{0.620pt}{0.118pt}}
\multiput(648.00,519.17)(11.713,10.000){2}{\rule{0.310pt}{0.400pt}}
\multiput(661.00,530.58)(0.652,0.491){17}{\rule{0.620pt}{0.118pt}}
\multiput(661.00,529.17)(11.713,10.000){2}{\rule{0.310pt}{0.400pt}}
\multiput(674.00,540.59)(0.786,0.489){15}{\rule{0.722pt}{0.118pt}}
\multiput(674.00,539.17)(12.501,9.000){2}{\rule{0.361pt}{0.400pt}}
\multiput(688.00,549.58)(0.652,0.491){17}{\rule{0.620pt}{0.118pt}}
\multiput(688.00,548.17)(11.713,10.000){2}{\rule{0.310pt}{0.400pt}}
\multiput(701.00,559.58)(0.704,0.491){17}{\rule{0.660pt}{0.118pt}}
\multiput(701.00,558.17)(12.630,10.000){2}{\rule{0.330pt}{0.400pt}}
\multiput(715.00,569.59)(0.728,0.489){15}{\rule{0.678pt}{0.118pt}}
\multiput(715.00,568.17)(11.593,9.000){2}{\rule{0.339pt}{0.400pt}}
\multiput(728.00,578.58)(0.652,0.491){17}{\rule{0.620pt}{0.118pt}}
\multiput(728.00,577.17)(11.713,10.000){2}{\rule{0.310pt}{0.400pt}}
\multiput(741.00,588.58)(0.704,0.491){17}{\rule{0.660pt}{0.118pt}}
\multiput(741.00,587.17)(12.630,10.000){2}{\rule{0.330pt}{0.400pt}}
\multiput(755.00,598.58)(0.652,0.491){17}{\rule{0.620pt}{0.118pt}}
\multiput(755.00,597.17)(11.713,10.000){2}{\rule{0.310pt}{0.400pt}}
\multiput(768.00,608.59)(0.786,0.489){15}{\rule{0.722pt}{0.118pt}}
\multiput(768.00,607.17)(12.501,9.000){2}{\rule{0.361pt}{0.400pt}}
\multiput(782.00,617.58)(0.652,0.491){17}{\rule{0.620pt}{0.118pt}}
\multiput(782.00,616.17)(11.713,10.000){2}{\rule{0.310pt}{0.400pt}}
\multiput(795.00,627.58)(0.704,0.491){17}{\rule{0.660pt}{0.118pt}}
\multiput(795.00,626.17)(12.630,10.000){2}{\rule{0.330pt}{0.400pt}}
\multiput(809.00,637.58)(0.652,0.491){17}{\rule{0.620pt}{0.118pt}}
\multiput(809.00,636.17)(11.713,10.000){2}{\rule{0.310pt}{0.400pt}}
\multiput(822.00,647.59)(0.728,0.489){15}{\rule{0.678pt}{0.118pt}}
\multiput(822.00,646.17)(11.593,9.000){2}{\rule{0.339pt}{0.400pt}}
\multiput(835.00,656.58)(0.704,0.491){17}{\rule{0.660pt}{0.118pt}}
\multiput(835.00,655.17)(12.630,10.000){2}{\rule{0.330pt}{0.400pt}}
\multiput(849.00,666.58)(0.652,0.491){17}{\rule{0.620pt}{0.118pt}}
\multiput(849.00,665.17)(11.713,10.000){2}{\rule{0.310pt}{0.400pt}}
\multiput(862.00,676.59)(0.786,0.489){15}{\rule{0.722pt}{0.118pt}}
\multiput(862.00,675.17)(12.501,9.000){2}{\rule{0.361pt}{0.400pt}}
\multiput(876.00,685.58)(0.652,0.491){17}{\rule{0.620pt}{0.118pt}}
\multiput(876.00,684.17)(11.713,10.000){2}{\rule{0.310pt}{0.400pt}}
\multiput(889.00,695.58)(0.652,0.491){17}{\rule{0.620pt}{0.118pt}}
\multiput(889.00,694.17)(11.713,10.000){2}{\rule{0.310pt}{0.400pt}}
\multiput(902.00,705.58)(0.704,0.491){17}{\rule{0.660pt}{0.118pt}}
\multiput(902.00,704.17)(12.630,10.000){2}{\rule{0.330pt}{0.400pt}}
\multiput(916.00,715.59)(0.728,0.489){15}{\rule{0.678pt}{0.118pt}}
\multiput(916.00,714.17)(11.593,9.000){2}{\rule{0.339pt}{0.400pt}}
\multiput(929.00,724.58)(0.704,0.491){17}{\rule{0.660pt}{0.118pt}}
\multiput(929.00,723.17)(12.630,10.000){2}{\rule{0.330pt}{0.400pt}}
\multiput(943.00,734.58)(0.652,0.491){17}{\rule{0.620pt}{0.118pt}}
\multiput(943.00,733.17)(11.713,10.000){2}{\rule{0.310pt}{0.400pt}}
\multiput(956.00,744.58)(0.704,0.491){17}{\rule{0.660pt}{0.118pt}}
\multiput(956.00,743.17)(12.630,10.000){2}{\rule{0.330pt}{0.400pt}}
\multiput(970.00,754.59)(0.728,0.489){15}{\rule{0.678pt}{0.118pt}}
\multiput(970.00,753.17)(11.593,9.000){2}{\rule{0.339pt}{0.400pt}}
\multiput(983.00,763.58)(0.652,0.491){17}{\rule{0.620pt}{0.118pt}}
\multiput(983.00,762.17)(11.713,10.000){2}{\rule{0.310pt}{0.400pt}}
\multiput(996.00,773.58)(0.704,0.491){17}{\rule{0.660pt}{0.118pt}}
\multiput(996.00,772.17)(12.630,10.000){2}{\rule{0.330pt}{0.400pt}}
\multiput(1010.00,783.59)(0.728,0.489){15}{\rule{0.678pt}{0.118pt}}
\multiput(1010.00,782.17)(11.593,9.000){2}{\rule{0.339pt}{0.400pt}}
\multiput(1023.00,792.58)(0.704,0.491){17}{\rule{0.660pt}{0.118pt}}
\multiput(1023.00,791.17)(12.630,10.000){2}{\rule{0.330pt}{0.400pt}}
\multiput(1037.00,802.58)(0.652,0.491){17}{\rule{0.620pt}{0.118pt}}
\multiput(1037.00,801.17)(11.713,10.000){2}{\rule{0.310pt}{0.400pt}}
\multiput(1050.00,812.58)(0.652,0.491){17}{\rule{0.620pt}{0.118pt}}
\multiput(1050.00,811.17)(11.713,10.000){2}{\rule{0.310pt}{0.400pt}}
\multiput(1063.00,822.59)(0.786,0.489){15}{\rule{0.722pt}{0.118pt}}
\multiput(1063.00,821.17)(12.501,9.000){2}{\rule{0.361pt}{0.400pt}}
\multiput(1077.00,831.58)(0.652,0.491){17}{\rule{0.620pt}{0.118pt}}
\multiput(1077.00,830.17)(11.713,10.000){2}{\rule{0.310pt}{0.400pt}}
\multiput(1090.00,841.58)(0.704,0.491){17}{\rule{0.660pt}{0.118pt}}
\multiput(1090.00,840.17)(12.630,10.000){2}{\rule{0.330pt}{0.400pt}}
\multiput(1104.00,851.59)(0.692,0.488){13}{\rule{0.650pt}{0.117pt}}
\multiput(1104.00,850.17)(9.651,8.000){2}{\rule{0.325pt}{0.400pt}}
\put(111,131){\usebox{\plotpoint}}
\put(111.00,131.00){\usebox{\plotpoint}}
\put(129.76,139.88){\usebox{\plotpoint}}
\put(148.48,148.84){\usebox{\plotpoint}}
\put(167.42,157.30){\usebox{\plotpoint}}
\put(185.93,166.66){\usebox{\plotpoint}}
\put(204.57,175.79){\usebox{\plotpoint}}
\put(223.48,184.35){\usebox{\plotpoint}}
\put(242.11,193.44){\usebox{\plotpoint}}
\put(261.04,201.94){\usebox{\plotpoint}}
\put(279.64,211.12){\usebox{\plotpoint}}
\put(298.49,219.78){\usebox{\plotpoint}}
\put(316.99,229.14){\usebox{\plotpoint}}
\put(335.95,237.59){\usebox{\plotpoint}}
\put(354.42,247.04){\usebox{\plotpoint}}
\put(373.40,255.42){\usebox{\plotpoint}}
\put(392.05,264.52){\usebox{\plotpoint}}
\put(410.81,273.40){\usebox{\plotpoint}}
\put(429.51,282.39){\usebox{\plotpoint}}
\put(448.32,291.16){\usebox{\plotpoint}}
\put(466.99,300.23){\usebox{\plotpoint}}
\put(486.00,308.57){\usebox{\plotpoint}}
\put(504.45,318.05){\usebox{\plotpoint}}
\put(523.14,327.07){\usebox{\plotpoint}}
\put(541.95,335.84){\usebox{\plotpoint}}
\put(560.73,344.62){\usebox{\plotpoint}}
\put(579.38,353.71){\usebox{\plotpoint}}
\put(598.26,362.30){\usebox{\plotpoint}}
\put(616.96,371.27){\usebox{\plotpoint}}
\put(635.46,380.63){\usebox{\plotpoint}}
\put(654.46,388.98){\usebox{\plotpoint}}
\put(672.93,398.43){\usebox{\plotpoint}}
\put(691.91,406.81){\usebox{\plotpoint}}
\put(710.61,415.81){\usebox{\plotpoint}}
\put(729.35,424.73){\usebox{\plotpoint}}
\put(747.92,433.96){\usebox{\plotpoint}}
\put(766.85,442.47){\usebox{\plotpoint}}
\put(785.48,451.61){\usebox{\plotpoint}}
\put(804.44,460.05){\usebox{\plotpoint}}
\put(822.94,469.43){\usebox{\plotpoint}}
\put(841.68,478.34){\usebox{\plotpoint}}
\put(860.41,487.27){\usebox{\plotpoint}}
\put(879.33,495.79){\usebox{\plotpoint}}
\put(897.87,505.09){\usebox{\plotpoint}}
\put(916.86,513.46){\usebox{\plotpoint}}
\put(935.40,522.74){\usebox{\plotpoint}}
\put(954.34,531.23){\usebox{\plotpoint}}
\put(972.97,540.37){\usebox{\plotpoint}}
\put(991.55,549.60){\usebox{\plotpoint}}
\put(1010.43,558.20){\usebox{\plotpoint}}
\put(1029.18,567.09){\usebox{\plotpoint}}
\put(1047.91,576.03){\usebox{\plotpoint}}
\put(1066.70,584.85){\usebox{\plotpoint}}
\put(1085.38,593.87){\usebox{\plotpoint}}
\put(1104.02,603.01){\usebox{\plotpoint}}
\put(1122.86,611.71){\usebox{\plotpoint}}
\put(1141.53,620.77){\usebox{\plotpoint}}
\put(1160.38,629.45){\usebox{\plotpoint}}
\put(1179.10,638.36){\usebox{\plotpoint}}
\put(1197.97,646.99){\usebox{\plotpoint}}
\put(1216.41,656.50){\usebox{\plotpoint}}
\put(1235.39,664.88){\usebox{\plotpoint}}
\put(1253.90,674.24){\usebox{\plotpoint}}
\put(1272.88,682.64){\usebox{\plotpoint}}
\put(1291.32,692.14){\usebox{\plotpoint}}
\put(1310.17,700.78){\usebox{\plotpoint}}
\put(1328.90,709.67){\usebox{\plotpoint}}
\put(1347.74,718.37){\usebox{\plotpoint}}
\put(1366.42,727.42){\usebox{\plotpoint}}
\put(1385.26,736.13){\usebox{\plotpoint}}
\put(1403.90,745.26){\usebox{\plotpoint}}
\put(1422.58,754.29){\usebox{\plotpoint}}
\put(1439,762){\usebox{\plotpoint}}
\sbox{\plotpoint}{\rule[-0.400pt]{0.800pt}{0.800pt}}%
\put(111,131){\usebox{\plotpoint}}
\multiput(111.00,132.38)(1.768,0.560){3}{\rule{2.280pt}{0.135pt}}
\multiput(111.00,129.34)(8.268,5.000){2}{\rule{1.140pt}{0.800pt}}
\put(124,136.34){\rule{3.000pt}{0.800pt}}
\multiput(124.00,134.34)(7.773,4.000){2}{\rule{1.500pt}{0.800pt}}
\multiput(138.00,141.38)(1.768,0.560){3}{\rule{2.280pt}{0.135pt}}
\multiput(138.00,138.34)(8.268,5.000){2}{\rule{1.140pt}{0.800pt}}
\multiput(151.00,146.38)(1.936,0.560){3}{\rule{2.440pt}{0.135pt}}
\multiput(151.00,143.34)(8.936,5.000){2}{\rule{1.220pt}{0.800pt}}
\multiput(165.00,151.38)(1.768,0.560){3}{\rule{2.280pt}{0.135pt}}
\multiput(165.00,148.34)(8.268,5.000){2}{\rule{1.140pt}{0.800pt}}
\put(178,155.34){\rule{2.800pt}{0.800pt}}
\multiput(178.00,153.34)(7.188,4.000){2}{\rule{1.400pt}{0.800pt}}
\multiput(191.00,160.38)(1.936,0.560){3}{\rule{2.440pt}{0.135pt}}
\multiput(191.00,157.34)(8.936,5.000){2}{\rule{1.220pt}{0.800pt}}
\multiput(205.00,165.38)(1.768,0.560){3}{\rule{2.280pt}{0.135pt}}
\multiput(205.00,162.34)(8.268,5.000){2}{\rule{1.140pt}{0.800pt}}
\multiput(218.00,170.38)(1.936,0.560){3}{\rule{2.440pt}{0.135pt}}
\multiput(218.00,167.34)(8.936,5.000){2}{\rule{1.220pt}{0.800pt}}
\put(232,174.34){\rule{2.800pt}{0.800pt}}
\multiput(232.00,172.34)(7.188,4.000){2}{\rule{1.400pt}{0.800pt}}
\multiput(245.00,179.38)(1.936,0.560){3}{\rule{2.440pt}{0.135pt}}
\multiput(245.00,176.34)(8.936,5.000){2}{\rule{1.220pt}{0.800pt}}
\multiput(259.00,184.38)(1.768,0.560){3}{\rule{2.280pt}{0.135pt}}
\multiput(259.00,181.34)(8.268,5.000){2}{\rule{1.140pt}{0.800pt}}
\multiput(272.00,189.38)(1.768,0.560){3}{\rule{2.280pt}{0.135pt}}
\multiput(272.00,186.34)(8.268,5.000){2}{\rule{1.140pt}{0.800pt}}
\put(285,193.34){\rule{3.000pt}{0.800pt}}
\multiput(285.00,191.34)(7.773,4.000){2}{\rule{1.500pt}{0.800pt}}
\multiput(299.00,198.38)(1.768,0.560){3}{\rule{2.280pt}{0.135pt}}
\multiput(299.00,195.34)(8.268,5.000){2}{\rule{1.140pt}{0.800pt}}
\multiput(312.00,203.38)(1.936,0.560){3}{\rule{2.440pt}{0.135pt}}
\multiput(312.00,200.34)(8.936,5.000){2}{\rule{1.220pt}{0.800pt}}
\multiput(326.00,208.38)(1.768,0.560){3}{\rule{2.280pt}{0.135pt}}
\multiput(326.00,205.34)(8.268,5.000){2}{\rule{1.140pt}{0.800pt}}
\put(339,212.34){\rule{2.800pt}{0.800pt}}
\multiput(339.00,210.34)(7.188,4.000){2}{\rule{1.400pt}{0.800pt}}
\multiput(352.00,217.38)(1.936,0.560){3}{\rule{2.440pt}{0.135pt}}
\multiput(352.00,214.34)(8.936,5.000){2}{\rule{1.220pt}{0.800pt}}
\multiput(366.00,222.38)(1.768,0.560){3}{\rule{2.280pt}{0.135pt}}
\multiput(366.00,219.34)(8.268,5.000){2}{\rule{1.140pt}{0.800pt}}
\multiput(379.00,227.38)(1.936,0.560){3}{\rule{2.440pt}{0.135pt}}
\multiput(379.00,224.34)(8.936,5.000){2}{\rule{1.220pt}{0.800pt}}
\put(393,231.34){\rule{2.800pt}{0.800pt}}
\multiput(393.00,229.34)(7.188,4.000){2}{\rule{1.400pt}{0.800pt}}
\multiput(406.00,236.38)(1.936,0.560){3}{\rule{2.440pt}{0.135pt}}
\multiput(406.00,233.34)(8.936,5.000){2}{\rule{1.220pt}{0.800pt}}
\multiput(420.00,241.38)(1.768,0.560){3}{\rule{2.280pt}{0.135pt}}
\multiput(420.00,238.34)(8.268,5.000){2}{\rule{1.140pt}{0.800pt}}
\put(433,245.34){\rule{2.800pt}{0.800pt}}
\multiput(433.00,243.34)(7.188,4.000){2}{\rule{1.400pt}{0.800pt}}
\multiput(446.00,250.38)(1.936,0.560){3}{\rule{2.440pt}{0.135pt}}
\multiput(446.00,247.34)(8.936,5.000){2}{\rule{1.220pt}{0.800pt}}
\multiput(460.00,255.38)(1.768,0.560){3}{\rule{2.280pt}{0.135pt}}
\multiput(460.00,252.34)(8.268,5.000){2}{\rule{1.140pt}{0.800pt}}
\multiput(473.00,260.38)(1.936,0.560){3}{\rule{2.440pt}{0.135pt}}
\multiput(473.00,257.34)(8.936,5.000){2}{\rule{1.220pt}{0.800pt}}
\put(487,264.34){\rule{2.800pt}{0.800pt}}
\multiput(487.00,262.34)(7.188,4.000){2}{\rule{1.400pt}{0.800pt}}
\multiput(500.00,269.38)(1.768,0.560){3}{\rule{2.280pt}{0.135pt}}
\multiput(500.00,266.34)(8.268,5.000){2}{\rule{1.140pt}{0.800pt}}
\multiput(513.00,274.38)(1.936,0.560){3}{\rule{2.440pt}{0.135pt}}
\multiput(513.00,271.34)(8.936,5.000){2}{\rule{1.220pt}{0.800pt}}
\multiput(527.00,279.38)(1.768,0.560){3}{\rule{2.280pt}{0.135pt}}
\multiput(527.00,276.34)(8.268,5.000){2}{\rule{1.140pt}{0.800pt}}
\put(540,283.34){\rule{3.000pt}{0.800pt}}
\multiput(540.00,281.34)(7.773,4.000){2}{\rule{1.500pt}{0.800pt}}
\multiput(554.00,288.38)(1.768,0.560){3}{\rule{2.280pt}{0.135pt}}
\multiput(554.00,285.34)(8.268,5.000){2}{\rule{1.140pt}{0.800pt}}
\multiput(567.00,293.38)(1.768,0.560){3}{\rule{2.280pt}{0.135pt}}
\multiput(567.00,290.34)(8.268,5.000){2}{\rule{1.140pt}{0.800pt}}
\multiput(580.00,298.38)(1.936,0.560){3}{\rule{2.440pt}{0.135pt}}
\multiput(580.00,295.34)(8.936,5.000){2}{\rule{1.220pt}{0.800pt}}
\put(594,302.34){\rule{2.800pt}{0.800pt}}
\multiput(594.00,300.34)(7.188,4.000){2}{\rule{1.400pt}{0.800pt}}
\multiput(607.00,307.38)(1.936,0.560){3}{\rule{2.440pt}{0.135pt}}
\multiput(607.00,304.34)(8.936,5.000){2}{\rule{1.220pt}{0.800pt}}
\multiput(621.00,312.38)(1.768,0.560){3}{\rule{2.280pt}{0.135pt}}
\multiput(621.00,309.34)(8.268,5.000){2}{\rule{1.140pt}{0.800pt}}
\multiput(634.00,317.38)(1.936,0.560){3}{\rule{2.440pt}{0.135pt}}
\multiput(634.00,314.34)(8.936,5.000){2}{\rule{1.220pt}{0.800pt}}
\put(648,321.34){\rule{2.800pt}{0.800pt}}
\multiput(648.00,319.34)(7.188,4.000){2}{\rule{1.400pt}{0.800pt}}
\multiput(661.00,326.38)(1.768,0.560){3}{\rule{2.280pt}{0.135pt}}
\multiput(661.00,323.34)(8.268,5.000){2}{\rule{1.140pt}{0.800pt}}
\multiput(674.00,331.38)(1.936,0.560){3}{\rule{2.440pt}{0.135pt}}
\multiput(674.00,328.34)(8.936,5.000){2}{\rule{1.220pt}{0.800pt}}
\multiput(688.00,336.38)(1.768,0.560){3}{\rule{2.280pt}{0.135pt}}
\multiput(688.00,333.34)(8.268,5.000){2}{\rule{1.140pt}{0.800pt}}
\put(701,340.34){\rule{3.000pt}{0.800pt}}
\multiput(701.00,338.34)(7.773,4.000){2}{\rule{1.500pt}{0.800pt}}
\multiput(715.00,345.38)(1.768,0.560){3}{\rule{2.280pt}{0.135pt}}
\multiput(715.00,342.34)(8.268,5.000){2}{\rule{1.140pt}{0.800pt}}
\multiput(728.00,350.38)(1.768,0.560){3}{\rule{2.280pt}{0.135pt}}
\multiput(728.00,347.34)(8.268,5.000){2}{\rule{1.140pt}{0.800pt}}
\put(741,354.34){\rule{3.000pt}{0.800pt}}
\multiput(741.00,352.34)(7.773,4.000){2}{\rule{1.500pt}{0.800pt}}
\multiput(755.00,359.38)(1.768,0.560){3}{\rule{2.280pt}{0.135pt}}
\multiput(755.00,356.34)(8.268,5.000){2}{\rule{1.140pt}{0.800pt}}
\multiput(768.00,364.38)(1.936,0.560){3}{\rule{2.440pt}{0.135pt}}
\multiput(768.00,361.34)(8.936,5.000){2}{\rule{1.220pt}{0.800pt}}
\multiput(782.00,369.38)(1.768,0.560){3}{\rule{2.280pt}{0.135pt}}
\multiput(782.00,366.34)(8.268,5.000){2}{\rule{1.140pt}{0.800pt}}
\put(795,373.34){\rule{3.000pt}{0.800pt}}
\multiput(795.00,371.34)(7.773,4.000){2}{\rule{1.500pt}{0.800pt}}
\multiput(809.00,378.38)(1.768,0.560){3}{\rule{2.280pt}{0.135pt}}
\multiput(809.00,375.34)(8.268,5.000){2}{\rule{1.140pt}{0.800pt}}
\multiput(822.00,383.38)(1.768,0.560){3}{\rule{2.280pt}{0.135pt}}
\multiput(822.00,380.34)(8.268,5.000){2}{\rule{1.140pt}{0.800pt}}
\multiput(835.00,388.38)(1.936,0.560){3}{\rule{2.440pt}{0.135pt}}
\multiput(835.00,385.34)(8.936,5.000){2}{\rule{1.220pt}{0.800pt}}
\put(849,392.34){\rule{2.800pt}{0.800pt}}
\multiput(849.00,390.34)(7.188,4.000){2}{\rule{1.400pt}{0.800pt}}
\multiput(862.00,397.38)(1.936,0.560){3}{\rule{2.440pt}{0.135pt}}
\multiput(862.00,394.34)(8.936,5.000){2}{\rule{1.220pt}{0.800pt}}
\multiput(876.00,402.38)(1.768,0.560){3}{\rule{2.280pt}{0.135pt}}
\multiput(876.00,399.34)(8.268,5.000){2}{\rule{1.140pt}{0.800pt}}
\multiput(889.00,407.38)(1.768,0.560){3}{\rule{2.280pt}{0.135pt}}
\multiput(889.00,404.34)(8.268,5.000){2}{\rule{1.140pt}{0.800pt}}
\put(902,411.34){\rule{3.000pt}{0.800pt}}
\multiput(902.00,409.34)(7.773,4.000){2}{\rule{1.500pt}{0.800pt}}
\multiput(916.00,416.38)(1.768,0.560){3}{\rule{2.280pt}{0.135pt}}
\multiput(916.00,413.34)(8.268,5.000){2}{\rule{1.140pt}{0.800pt}}
\multiput(929.00,421.38)(1.936,0.560){3}{\rule{2.440pt}{0.135pt}}
\multiput(929.00,418.34)(8.936,5.000){2}{\rule{1.220pt}{0.800pt}}
\multiput(943.00,426.38)(1.768,0.560){3}{\rule{2.280pt}{0.135pt}}
\multiput(943.00,423.34)(8.268,5.000){2}{\rule{1.140pt}{0.800pt}}
\put(956,430.34){\rule{3.000pt}{0.800pt}}
\multiput(956.00,428.34)(7.773,4.000){2}{\rule{1.500pt}{0.800pt}}
\multiput(970.00,435.38)(1.768,0.560){3}{\rule{2.280pt}{0.135pt}}
\multiput(970.00,432.34)(8.268,5.000){2}{\rule{1.140pt}{0.800pt}}
\multiput(983.00,440.38)(1.768,0.560){3}{\rule{2.280pt}{0.135pt}}
\multiput(983.00,437.34)(8.268,5.000){2}{\rule{1.140pt}{0.800pt}}
\multiput(996.00,445.38)(1.936,0.560){3}{\rule{2.440pt}{0.135pt}}
\multiput(996.00,442.34)(8.936,5.000){2}{\rule{1.220pt}{0.800pt}}
\put(1010,449.34){\rule{2.800pt}{0.800pt}}
\multiput(1010.00,447.34)(7.188,4.000){2}{\rule{1.400pt}{0.800pt}}
\multiput(1023.00,454.38)(1.936,0.560){3}{\rule{2.440pt}{0.135pt}}
\multiput(1023.00,451.34)(8.936,5.000){2}{\rule{1.220pt}{0.800pt}}
\multiput(1037.00,459.38)(1.768,0.560){3}{\rule{2.280pt}{0.135pt}}
\multiput(1037.00,456.34)(8.268,5.000){2}{\rule{1.140pt}{0.800pt}}
\put(1050,463.34){\rule{2.800pt}{0.800pt}}
\multiput(1050.00,461.34)(7.188,4.000){2}{\rule{1.400pt}{0.800pt}}
\multiput(1063.00,468.38)(1.936,0.560){3}{\rule{2.440pt}{0.135pt}}
\multiput(1063.00,465.34)(8.936,5.000){2}{\rule{1.220pt}{0.800pt}}
\multiput(1077.00,473.38)(1.768,0.560){3}{\rule{2.280pt}{0.135pt}}
\multiput(1077.00,470.34)(8.268,5.000){2}{\rule{1.140pt}{0.800pt}}
\multiput(1090.00,478.38)(1.936,0.560){3}{\rule{2.440pt}{0.135pt}}
\multiput(1090.00,475.34)(8.936,5.000){2}{\rule{1.220pt}{0.800pt}}
\put(1104,482.34){\rule{2.800pt}{0.800pt}}
\multiput(1104.00,480.34)(7.188,4.000){2}{\rule{1.400pt}{0.800pt}}
\multiput(1117.00,487.38)(1.768,0.560){3}{\rule{2.280pt}{0.135pt}}
\multiput(1117.00,484.34)(8.268,5.000){2}{\rule{1.140pt}{0.800pt}}
\multiput(1130.00,492.38)(1.936,0.560){3}{\rule{2.440pt}{0.135pt}}
\multiput(1130.00,489.34)(8.936,5.000){2}{\rule{1.220pt}{0.800pt}}
\multiput(1144.00,497.38)(1.768,0.560){3}{\rule{2.280pt}{0.135pt}}
\multiput(1144.00,494.34)(8.268,5.000){2}{\rule{1.140pt}{0.800pt}}
\put(1157,501.34){\rule{3.000pt}{0.800pt}}
\multiput(1157.00,499.34)(7.773,4.000){2}{\rule{1.500pt}{0.800pt}}
\multiput(1171.00,506.38)(1.768,0.560){3}{\rule{2.280pt}{0.135pt}}
\multiput(1171.00,503.34)(8.268,5.000){2}{\rule{1.140pt}{0.800pt}}
\multiput(1184.00,511.38)(1.936,0.560){3}{\rule{2.440pt}{0.135pt}}
\multiput(1184.00,508.34)(8.936,5.000){2}{\rule{1.220pt}{0.800pt}}
\multiput(1198.00,516.38)(1.768,0.560){3}{\rule{2.280pt}{0.135pt}}
\multiput(1198.00,513.34)(8.268,5.000){2}{\rule{1.140pt}{0.800pt}}
\put(1211,520.34){\rule{2.800pt}{0.800pt}}
\multiput(1211.00,518.34)(7.188,4.000){2}{\rule{1.400pt}{0.800pt}}
\multiput(1224.00,525.38)(1.936,0.560){3}{\rule{2.440pt}{0.135pt}}
\multiput(1224.00,522.34)(8.936,5.000){2}{\rule{1.220pt}{0.800pt}}
\multiput(1238.00,530.38)(1.768,0.560){3}{\rule{2.280pt}{0.135pt}}
\multiput(1238.00,527.34)(8.268,5.000){2}{\rule{1.140pt}{0.800pt}}
\multiput(1251.00,535.38)(1.936,0.560){3}{\rule{2.440pt}{0.135pt}}
\multiput(1251.00,532.34)(8.936,5.000){2}{\rule{1.220pt}{0.800pt}}
\put(1265,539.34){\rule{2.800pt}{0.800pt}}
\multiput(1265.00,537.34)(7.188,4.000){2}{\rule{1.400pt}{0.800pt}}
\multiput(1278.00,544.38)(1.768,0.560){3}{\rule{2.280pt}{0.135pt}}
\multiput(1278.00,541.34)(8.268,5.000){2}{\rule{1.140pt}{0.800pt}}
\multiput(1291.00,549.38)(1.936,0.560){3}{\rule{2.440pt}{0.135pt}}
\multiput(1291.00,546.34)(8.936,5.000){2}{\rule{1.220pt}{0.800pt}}
\put(1305,553.34){\rule{2.800pt}{0.800pt}}
\multiput(1305.00,551.34)(7.188,4.000){2}{\rule{1.400pt}{0.800pt}}
\multiput(1318.00,558.38)(1.936,0.560){3}{\rule{2.440pt}{0.135pt}}
\multiput(1318.00,555.34)(8.936,5.000){2}{\rule{1.220pt}{0.800pt}}
\multiput(1332.00,563.38)(1.768,0.560){3}{\rule{2.280pt}{0.135pt}}
\multiput(1332.00,560.34)(8.268,5.000){2}{\rule{1.140pt}{0.800pt}}
\multiput(1345.00,568.38)(1.936,0.560){3}{\rule{2.440pt}{0.135pt}}
\multiput(1345.00,565.34)(8.936,5.000){2}{\rule{1.220pt}{0.800pt}}
\put(1359,572.34){\rule{2.800pt}{0.800pt}}
\multiput(1359.00,570.34)(7.188,4.000){2}{\rule{1.400pt}{0.800pt}}
\multiput(1372.00,577.38)(1.768,0.560){3}{\rule{2.280pt}{0.135pt}}
\multiput(1372.00,574.34)(8.268,5.000){2}{\rule{1.140pt}{0.800pt}}
\multiput(1385.00,582.38)(1.936,0.560){3}{\rule{2.440pt}{0.135pt}}
\multiput(1385.00,579.34)(8.936,5.000){2}{\rule{1.220pt}{0.800pt}}
\multiput(1399.00,587.38)(1.768,0.560){3}{\rule{2.280pt}{0.135pt}}
\multiput(1399.00,584.34)(8.268,5.000){2}{\rule{1.140pt}{0.800pt}}
\put(1412,591.34){\rule{3.000pt}{0.800pt}}
\multiput(1412.00,589.34)(7.773,4.000){2}{\rule{1.500pt}{0.800pt}}
\multiput(1426.00,596.38)(1.768,0.560){3}{\rule{2.280pt}{0.135pt}}
\multiput(1426.00,593.34)(8.268,5.000){2}{\rule{1.140pt}{0.800pt}}
\sbox{\plotpoint}{\rule[-0.500pt]{1.000pt}{1.000pt}}%
\put(111,131){\usebox{\plotpoint}}
\put(111.00,131.00){\usebox{\plotpoint}}
\put(130.88,136.97){\usebox{\plotpoint}}
\put(151.01,142.00){\usebox{\plotpoint}}
\put(170.93,147.82){\usebox{\plotpoint}}
\put(190.77,153.93){\usebox{\plotpoint}}
\put(210.92,158.82){\usebox{\plotpoint}}
\put(230.83,164.67){\usebox{\plotpoint}}
\put(250.81,170.24){\usebox{\plotpoint}}
\put(270.83,175.64){\usebox{\plotpoint}}
\put(290.70,181.63){\usebox{\plotpoint}}
\put(310.59,187.57){\usebox{\plotpoint}}
\put(330.74,192.46){\usebox{\plotpoint}}
\put(350.58,198.56){\usebox{\plotpoint}}
\put(370.59,204.06){\usebox{\plotpoint}}
\put(390.66,209.33){\usebox{\plotpoint}}
\put(410.54,215.30){\usebox{\plotpoint}}
\put(430.43,221.21){\usebox{\plotpoint}}
\put(450.54,226.30){\usebox{\plotpoint}}
\put(470.44,232.21){\usebox{\plotpoint}}
\put(490.42,237.79){\usebox{\plotpoint}}
\put(510.45,243.21){\usebox{\plotpoint}}
\put(530.37,249.04){\usebox{\plotpoint}}
\put(550.44,254.24){\usebox{\plotpoint}}
\put(570.36,260.03){\usebox{\plotpoint}}
\put(590.26,265.93){\usebox{\plotpoint}}
\put(610.19,271.68){\usebox{\plotpoint}}
\put(630.27,276.85){\usebox{\plotpoint}}
\put(650.19,282.67){\usebox{\plotpoint}}
\put(670.20,288.12){\usebox{\plotpoint}}
\put(690.20,293.68){\usebox{\plotpoint}}
\put(710.09,299.60){\usebox{\plotpoint}}
\put(730.21,304.68){\usebox{\plotpoint}}
\put(750.10,310.60){\usebox{\plotpoint}}
\put(769.98,316.56){\usebox{\plotpoint}}
\put(790.04,321.86){\usebox{\plotpoint}}
\put(810.06,327.32){\usebox{\plotpoint}}
\put(829.89,333.43){\usebox{\plotpoint}}
\put(850.05,338.32){\usebox{\plotpoint}}
\put(869.93,344.27){\usebox{\plotpoint}}
\put(889.80,350.25){\usebox{\plotpoint}}
\put(909.82,355.68){\usebox{\plotpoint}}
\put(929.80,361.23){\usebox{\plotpoint}}
\put(949.72,367.07){\usebox{\plotpoint}}
\put(969.87,371.97){\usebox{\plotpoint}}
\put(989.71,378.06){\usebox{\plotpoint}}
\put(1009.63,383.89){\usebox{\plotpoint}}
\put(1029.76,388.93){\usebox{\plotpoint}}
\put(1049.64,394.89){\usebox{\plotpoint}}
\put(1069.51,400.86){\usebox{\plotpoint}}
\put(1089.64,405.92){\usebox{\plotpoint}}
\put(1109.56,411.71){\usebox{\plotpoint}}
\put(1129.40,417.82){\usebox{\plotpoint}}
\put(1149.56,422.71){\usebox{\plotpoint}}
\put(1169.47,428.56){\usebox{\plotpoint}}
\put(1189.35,434.53){\usebox{\plotpoint}}
\put(1209.45,439.64){\usebox{\plotpoint}}
\put(1229.35,445.53){\usebox{\plotpoint}}
\put(1249.24,451.46){\usebox{\plotpoint}}
\put(1269.40,456.35){\usebox{\plotpoint}}
\put(1289.23,462.46){\usebox{\plotpoint}}
\put(1309.24,467.98){\usebox{\plotpoint}}
\put(1329.31,473.23){\usebox{\plotpoint}}
\put(1349.19,479.20){\usebox{\plotpoint}}
\put(1369.08,485.10){\usebox{\plotpoint}}
\put(1389.19,490.20){\usebox{\plotpoint}}
\put(1409.09,496.10){\usebox{\plotpoint}}
\put(1429.07,501.71){\usebox{\plotpoint}}
\put(1439,504){\usebox{\plotpoint}}
\sbox{\plotpoint}{\rule[-0.200pt]{0.400pt}{0.400pt}}%
\put(111.0,131.0){\rule[-0.200pt]{0.400pt}{175.375pt}}
\put(111.0,131.0){\rule[-0.200pt]{319.915pt}{0.400pt}}
\put(1439.0,131.0){\rule[-0.200pt]{0.400pt}{175.375pt}}
\put(111.0,859.0){\rule[-0.200pt]{319.915pt}{0.400pt}}
\end{picture}
}
    \caption{Sistema B}
  \end{subfigure}
  \begin{subfigure}[b]{0.3\textwidth}
    \scalebox{.4}{% GNUPLOT: LaTeX picture
\setlength{\unitlength}{0.240900pt}
\ifx\plotpoint\undefined\newsavebox{\plotpoint}\fi
\sbox{\plotpoint}{\rule[-0.200pt]{0.400pt}{0.400pt}}%
\begin{picture}(1500,900)(0,0)
\sbox{\plotpoint}{\rule[-0.200pt]{0.400pt}{0.400pt}}%
\put(151.0,131.0){\rule[-0.200pt]{4.818pt}{0.400pt}}
\put(131,131){\makebox(0,0)[r]{$0$}}
\put(1419.0,131.0){\rule[-0.200pt]{4.818pt}{0.400pt}}
\put(151.0,277.0){\rule[-0.200pt]{4.818pt}{0.400pt}}
\put(131,277){\makebox(0,0)[r]{$20$}}
\put(1419.0,277.0){\rule[-0.200pt]{4.818pt}{0.400pt}}
\put(151.0,422.0){\rule[-0.200pt]{4.818pt}{0.400pt}}
\put(131,422){\makebox(0,0)[r]{$40$}}
\put(1419.0,422.0){\rule[-0.200pt]{4.818pt}{0.400pt}}
\put(151.0,568.0){\rule[-0.200pt]{4.818pt}{0.400pt}}
\put(131,568){\makebox(0,0)[r]{$60$}}
\put(1419.0,568.0){\rule[-0.200pt]{4.818pt}{0.400pt}}
\put(151.0,713.0){\rule[-0.200pt]{4.818pt}{0.400pt}}
\put(131,713){\makebox(0,0)[r]{$80$}}
\put(1419.0,713.0){\rule[-0.200pt]{4.818pt}{0.400pt}}
\put(151.0,859.0){\rule[-0.200pt]{4.818pt}{0.400pt}}
\put(131,859){\makebox(0,0)[r]{$100$}}
\put(1419.0,859.0){\rule[-0.200pt]{4.818pt}{0.400pt}}
\put(151.0,131.0){\rule[-0.200pt]{0.400pt}{4.818pt}}
\put(151,90){\makebox(0,0){$0$}}
\put(151.0,839.0){\rule[-0.200pt]{0.400pt}{4.818pt}}
\put(409.0,131.0){\rule[-0.200pt]{0.400pt}{4.818pt}}
\put(409,90){\makebox(0,0){$2$}}
\put(409.0,839.0){\rule[-0.200pt]{0.400pt}{4.818pt}}
\put(666.0,131.0){\rule[-0.200pt]{0.400pt}{4.818pt}}
\put(666,90){\makebox(0,0){$4$}}
\put(666.0,839.0){\rule[-0.200pt]{0.400pt}{4.818pt}}
\put(924.0,131.0){\rule[-0.200pt]{0.400pt}{4.818pt}}
\put(924,90){\makebox(0,0){$6$}}
\put(924.0,839.0){\rule[-0.200pt]{0.400pt}{4.818pt}}
\put(1181.0,131.0){\rule[-0.200pt]{0.400pt}{4.818pt}}
\put(1181,90){\makebox(0,0){$8$}}
\put(1181.0,839.0){\rule[-0.200pt]{0.400pt}{4.818pt}}
\put(1439.0,131.0){\rule[-0.200pt]{0.400pt}{4.818pt}}
\put(1439,90){\makebox(0,0){$10$}}
\put(1439.0,839.0){\rule[-0.200pt]{0.400pt}{4.818pt}}
\put(151.0,131.0){\rule[-0.200pt]{0.400pt}{175.375pt}}
\put(151.0,131.0){\rule[-0.200pt]{310.279pt}{0.400pt}}
\put(1439.0,131.0){\rule[-0.200pt]{0.400pt}{175.375pt}}
\put(151.0,859.0){\rule[-0.200pt]{310.279pt}{0.400pt}}
\put(924,713){\makebox(0,0)[l]{$P=\frac{T_a}{V}$}}
\put(30,495){\makebox(0,0){$P$}}
\put(795,29){\makebox(0,0){$V$}}
\multiput(185.59,816.41)(0.477,-14.068){7}{\rule{0.115pt}{10.260pt}}
\multiput(184.17,837.70)(5.000,-105.705){2}{\rule{0.400pt}{5.130pt}}
\multiput(190.58,712.30)(0.493,-5.968){23}{\rule{0.119pt}{4.746pt}}
\multiput(189.17,722.15)(13.000,-141.149){2}{\rule{0.400pt}{2.373pt}}
\multiput(203.58,569.09)(0.493,-3.550){23}{\rule{0.119pt}{2.869pt}}
\multiput(202.17,575.04)(13.000,-84.045){2}{\rule{0.400pt}{1.435pt}}
\multiput(216.58,482.92)(0.493,-2.360){23}{\rule{0.119pt}{1.946pt}}
\multiput(215.17,486.96)(13.000,-55.961){2}{\rule{0.400pt}{0.973pt}}
\multiput(229.58,425.09)(0.493,-1.686){23}{\rule{0.119pt}{1.423pt}}
\multiput(228.17,428.05)(13.000,-40.046){2}{\rule{0.400pt}{0.712pt}}
\multiput(242.58,383.50)(0.493,-1.250){23}{\rule{0.119pt}{1.085pt}}
\multiput(241.17,385.75)(13.000,-29.749){2}{\rule{0.400pt}{0.542pt}}
\multiput(255.58,352.39)(0.493,-0.972){23}{\rule{0.119pt}{0.869pt}}
\multiput(254.17,354.20)(13.000,-23.196){2}{\rule{0.400pt}{0.435pt}}
\multiput(268.58,328.03)(0.493,-0.774){23}{\rule{0.119pt}{0.715pt}}
\multiput(267.17,329.52)(13.000,-18.515){2}{\rule{0.400pt}{0.358pt}}
\multiput(281.58,308.54)(0.493,-0.616){23}{\rule{0.119pt}{0.592pt}}
\multiput(280.17,309.77)(13.000,-14.771){2}{\rule{0.400pt}{0.296pt}}
\multiput(294.58,292.80)(0.493,-0.536){23}{\rule{0.119pt}{0.531pt}}
\multiput(293.17,293.90)(13.000,-12.898){2}{\rule{0.400pt}{0.265pt}}
\multiput(307.00,279.92)(0.590,-0.492){19}{\rule{0.573pt}{0.118pt}}
\multiput(307.00,280.17)(11.811,-11.000){2}{\rule{0.286pt}{0.400pt}}
\multiput(320.00,268.92)(0.652,-0.491){17}{\rule{0.620pt}{0.118pt}}
\multiput(320.00,269.17)(11.713,-10.000){2}{\rule{0.310pt}{0.400pt}}
\multiput(333.00,258.93)(0.728,-0.489){15}{\rule{0.678pt}{0.118pt}}
\multiput(333.00,259.17)(11.593,-9.000){2}{\rule{0.339pt}{0.400pt}}
\multiput(346.00,249.93)(0.950,-0.485){11}{\rule{0.843pt}{0.117pt}}
\multiput(346.00,250.17)(11.251,-7.000){2}{\rule{0.421pt}{0.400pt}}
\multiput(359.00,242.93)(0.950,-0.485){11}{\rule{0.843pt}{0.117pt}}
\multiput(359.00,243.17)(11.251,-7.000){2}{\rule{0.421pt}{0.400pt}}
\multiput(372.00,235.93)(1.123,-0.482){9}{\rule{0.967pt}{0.116pt}}
\multiput(372.00,236.17)(10.994,-6.000){2}{\rule{0.483pt}{0.400pt}}
\multiput(385.00,229.93)(1.378,-0.477){7}{\rule{1.140pt}{0.115pt}}
\multiput(385.00,230.17)(10.634,-5.000){2}{\rule{0.570pt}{0.400pt}}
\multiput(398.00,224.93)(1.378,-0.477){7}{\rule{1.140pt}{0.115pt}}
\multiput(398.00,225.17)(10.634,-5.000){2}{\rule{0.570pt}{0.400pt}}
\multiput(411.00,219.94)(1.797,-0.468){5}{\rule{1.400pt}{0.113pt}}
\multiput(411.00,220.17)(10.094,-4.000){2}{\rule{0.700pt}{0.400pt}}
\multiput(424.00,215.94)(1.797,-0.468){5}{\rule{1.400pt}{0.113pt}}
\multiput(424.00,216.17)(10.094,-4.000){2}{\rule{0.700pt}{0.400pt}}
\multiput(437.00,211.94)(1.797,-0.468){5}{\rule{1.400pt}{0.113pt}}
\multiput(437.00,212.17)(10.094,-4.000){2}{\rule{0.700pt}{0.400pt}}
\multiput(450.00,207.95)(2.695,-0.447){3}{\rule{1.833pt}{0.108pt}}
\multiput(450.00,208.17)(9.195,-3.000){2}{\rule{0.917pt}{0.400pt}}
\multiput(463.00,204.95)(2.695,-0.447){3}{\rule{1.833pt}{0.108pt}}
\multiput(463.00,205.17)(9.195,-3.000){2}{\rule{0.917pt}{0.400pt}}
\multiput(476.00,201.95)(2.695,-0.447){3}{\rule{1.833pt}{0.108pt}}
\multiput(476.00,202.17)(9.195,-3.000){2}{\rule{0.917pt}{0.400pt}}
\put(489,198.17){\rule{2.700pt}{0.400pt}}
\multiput(489.00,199.17)(7.396,-2.000){2}{\rule{1.350pt}{0.400pt}}
\multiput(502.00,196.95)(2.695,-0.447){3}{\rule{1.833pt}{0.108pt}}
\multiput(502.00,197.17)(9.195,-3.000){2}{\rule{0.917pt}{0.400pt}}
\put(515,193.17){\rule{2.700pt}{0.400pt}}
\multiput(515.00,194.17)(7.396,-2.000){2}{\rule{1.350pt}{0.400pt}}
\put(528,191.17){\rule{2.700pt}{0.400pt}}
\multiput(528.00,192.17)(7.396,-2.000){2}{\rule{1.350pt}{0.400pt}}
\put(541,189.17){\rule{2.700pt}{0.400pt}}
\multiput(541.00,190.17)(7.396,-2.000){2}{\rule{1.350pt}{0.400pt}}
\put(554,187.17){\rule{2.700pt}{0.400pt}}
\multiput(554.00,188.17)(7.396,-2.000){2}{\rule{1.350pt}{0.400pt}}
\put(567,185.67){\rule{3.132pt}{0.400pt}}
\multiput(567.00,186.17)(6.500,-1.000){2}{\rule{1.566pt}{0.400pt}}
\put(580,184.17){\rule{2.700pt}{0.400pt}}
\multiput(580.00,185.17)(7.396,-2.000){2}{\rule{1.350pt}{0.400pt}}
\put(593,182.17){\rule{2.700pt}{0.400pt}}
\multiput(593.00,183.17)(7.396,-2.000){2}{\rule{1.350pt}{0.400pt}}
\put(606,180.67){\rule{3.132pt}{0.400pt}}
\multiput(606.00,181.17)(6.500,-1.000){2}{\rule{1.566pt}{0.400pt}}
\put(619,179.67){\rule{3.132pt}{0.400pt}}
\multiput(619.00,180.17)(6.500,-1.000){2}{\rule{1.566pt}{0.400pt}}
\put(632,178.17){\rule{2.700pt}{0.400pt}}
\multiput(632.00,179.17)(7.396,-2.000){2}{\rule{1.350pt}{0.400pt}}
\put(645,176.67){\rule{3.132pt}{0.400pt}}
\multiput(645.00,177.17)(6.500,-1.000){2}{\rule{1.566pt}{0.400pt}}
\put(658,175.67){\rule{3.132pt}{0.400pt}}
\multiput(658.00,176.17)(6.500,-1.000){2}{\rule{1.566pt}{0.400pt}}
\put(671,174.67){\rule{3.132pt}{0.400pt}}
\multiput(671.00,175.17)(6.500,-1.000){2}{\rule{1.566pt}{0.400pt}}
\put(684,173.67){\rule{3.132pt}{0.400pt}}
\multiput(684.00,174.17)(6.500,-1.000){2}{\rule{1.566pt}{0.400pt}}
\put(697,172.67){\rule{3.132pt}{0.400pt}}
\multiput(697.00,173.17)(6.500,-1.000){2}{\rule{1.566pt}{0.400pt}}
\put(710,171.67){\rule{3.132pt}{0.400pt}}
\multiput(710.00,172.17)(6.500,-1.000){2}{\rule{1.566pt}{0.400pt}}
\put(723,170.67){\rule{3.132pt}{0.400pt}}
\multiput(723.00,171.17)(6.500,-1.000){2}{\rule{1.566pt}{0.400pt}}
\put(736,169.67){\rule{3.132pt}{0.400pt}}
\multiput(736.00,170.17)(6.500,-1.000){2}{\rule{1.566pt}{0.400pt}}
\put(749,168.67){\rule{3.132pt}{0.400pt}}
\multiput(749.00,169.17)(6.500,-1.000){2}{\rule{1.566pt}{0.400pt}}
\put(775,167.67){\rule{3.132pt}{0.400pt}}
\multiput(775.00,168.17)(6.500,-1.000){2}{\rule{1.566pt}{0.400pt}}
\put(788,166.67){\rule{3.373pt}{0.400pt}}
\multiput(788.00,167.17)(7.000,-1.000){2}{\rule{1.686pt}{0.400pt}}
\put(802,165.67){\rule{3.132pt}{0.400pt}}
\multiput(802.00,166.17)(6.500,-1.000){2}{\rule{1.566pt}{0.400pt}}
\put(762.0,169.0){\rule[-0.200pt]{3.132pt}{0.400pt}}
\put(828,164.67){\rule{3.132pt}{0.400pt}}
\multiput(828.00,165.17)(6.500,-1.000){2}{\rule{1.566pt}{0.400pt}}
\put(841,163.67){\rule{3.132pt}{0.400pt}}
\multiput(841.00,164.17)(6.500,-1.000){2}{\rule{1.566pt}{0.400pt}}
\put(815.0,166.0){\rule[-0.200pt]{3.132pt}{0.400pt}}
\put(867,162.67){\rule{3.132pt}{0.400pt}}
\multiput(867.00,163.17)(6.500,-1.000){2}{\rule{1.566pt}{0.400pt}}
\put(854.0,164.0){\rule[-0.200pt]{3.132pt}{0.400pt}}
\put(893,161.67){\rule{3.132pt}{0.400pt}}
\multiput(893.00,162.17)(6.500,-1.000){2}{\rule{1.566pt}{0.400pt}}
\put(880.0,163.0){\rule[-0.200pt]{3.132pt}{0.400pt}}
\put(919,160.67){\rule{3.132pt}{0.400pt}}
\multiput(919.00,161.17)(6.500,-1.000){2}{\rule{1.566pt}{0.400pt}}
\put(906.0,162.0){\rule[-0.200pt]{3.132pt}{0.400pt}}
\put(945,159.67){\rule{3.132pt}{0.400pt}}
\multiput(945.00,160.17)(6.500,-1.000){2}{\rule{1.566pt}{0.400pt}}
\put(932.0,161.0){\rule[-0.200pt]{3.132pt}{0.400pt}}
\put(971,158.67){\rule{3.132pt}{0.400pt}}
\multiput(971.00,159.17)(6.500,-1.000){2}{\rule{1.566pt}{0.400pt}}
\put(958.0,160.0){\rule[-0.200pt]{3.132pt}{0.400pt}}
\put(997,157.67){\rule{3.132pt}{0.400pt}}
\multiput(997.00,158.17)(6.500,-1.000){2}{\rule{1.566pt}{0.400pt}}
\put(984.0,159.0){\rule[-0.200pt]{3.132pt}{0.400pt}}
\put(1023,156.67){\rule{3.132pt}{0.400pt}}
\multiput(1023.00,157.17)(6.500,-1.000){2}{\rule{1.566pt}{0.400pt}}
\put(1010.0,158.0){\rule[-0.200pt]{3.132pt}{0.400pt}}
\put(1062,155.67){\rule{3.132pt}{0.400pt}}
\multiput(1062.00,156.17)(6.500,-1.000){2}{\rule{1.566pt}{0.400pt}}
\put(1036.0,157.0){\rule[-0.200pt]{6.263pt}{0.400pt}}
\put(1101,154.67){\rule{3.132pt}{0.400pt}}
\multiput(1101.00,155.17)(6.500,-1.000){2}{\rule{1.566pt}{0.400pt}}
\put(1075.0,156.0){\rule[-0.200pt]{6.263pt}{0.400pt}}
\put(1140,153.67){\rule{3.132pt}{0.400pt}}
\multiput(1140.00,154.17)(6.500,-1.000){2}{\rule{1.566pt}{0.400pt}}
\put(1114.0,155.0){\rule[-0.200pt]{6.263pt}{0.400pt}}
\put(1192,152.67){\rule{3.132pt}{0.400pt}}
\multiput(1192.00,153.17)(6.500,-1.000){2}{\rule{1.566pt}{0.400pt}}
\put(1153.0,154.0){\rule[-0.200pt]{9.395pt}{0.400pt}}
\put(1231,151.67){\rule{3.132pt}{0.400pt}}
\multiput(1231.00,152.17)(6.500,-1.000){2}{\rule{1.566pt}{0.400pt}}
\put(1205.0,153.0){\rule[-0.200pt]{6.263pt}{0.400pt}}
\put(1283,150.67){\rule{3.132pt}{0.400pt}}
\multiput(1283.00,151.17)(6.500,-1.000){2}{\rule{1.566pt}{0.400pt}}
\put(1244.0,152.0){\rule[-0.200pt]{9.395pt}{0.400pt}}
\put(1348,149.67){\rule{3.132pt}{0.400pt}}
\multiput(1348.00,150.17)(6.500,-1.000){2}{\rule{1.566pt}{0.400pt}}
\put(1296.0,151.0){\rule[-0.200pt]{12.527pt}{0.400pt}}
\put(1413,148.67){\rule{3.132pt}{0.400pt}}
\multiput(1413.00,149.17)(6.500,-1.000){2}{\rule{1.566pt}{0.400pt}}
\put(1361.0,150.0){\rule[-0.200pt]{12.527pt}{0.400pt}}
\put(1426.0,149.0){\rule[-0.200pt]{3.132pt}{0.400pt}}
\put(216.00,859.00){\usebox{\plotpoint}}
\multiput(216,852)(2.235,-20.635){6}{\usebox{\plotpoint}}
\multiput(229,732)(3.102,-20.522){4}{\usebox{\plotpoint}}
\multiput(242,646)(4.070,-20.352){3}{\usebox{\plotpoint}}
\multiput(255,581)(5.223,-20.088){3}{\usebox{\plotpoint}}
\multiput(268,531)(6.415,-19.739){2}{\usebox{\plotpoint}}
\put(288.40,472.78){\usebox{\plotpoint}}
\multiput(294,459)(8.740,-18.825){2}{\usebox{\plotpoint}}
\put(315.13,416.62){\usebox{\plotpoint}}
\put(325.92,398.90){\usebox{\plotpoint}}
\put(337.71,381.84){\usebox{\plotpoint}}
\put(350.66,365.62){\usebox{\plotpoint}}
\put(364.67,350.33){\usebox{\plotpoint}}
\put(379.64,335.95){\usebox{\plotpoint}}
\put(395.66,322.80){\usebox{\plotpoint}}
\put(412.20,310.26){\usebox{\plotpoint}}
\put(429.88,299.38){\usebox{\plotpoint}}
\put(447.91,289.13){\usebox{\plotpoint}}
\put(466.28,279.48){\usebox{\plotpoint}}
\put(485.38,271.39){\usebox{\plotpoint}}
\put(504.45,263.25){\usebox{\plotpoint}}
\put(524.07,256.51){\usebox{\plotpoint}}
\put(543.81,250.13){\usebox{\plotpoint}}
\put(563.84,244.73){\usebox{\plotpoint}}
\put(583.81,239.12){\usebox{\plotpoint}}
\put(604.03,234.45){\usebox{\plotpoint}}
\put(624.26,229.79){\usebox{\plotpoint}}
\put(644.66,226.05){\usebox{\plotpoint}}
\put(664.99,221.93){\usebox{\plotpoint}}
\put(685.50,218.77){\usebox{\plotpoint}}
\put(706.02,215.61){\usebox{\plotpoint}}
\put(726.53,212.46){\usebox{\plotpoint}}
\put(747.04,209.30){\usebox{\plotpoint}}
\put(767.67,207.13){\usebox{\plotpoint}}
\put(788.30,204.96){\usebox{\plotpoint}}
\put(808.90,202.47){\usebox{\plotpoint}}
\put(829.48,199.89){\usebox{\plotpoint}}
\put(850.17,198.29){\usebox{\plotpoint}}
\put(870.83,196.41){\usebox{\plotpoint}}
\put(891.44,194.12){\usebox{\plotpoint}}
\put(912.14,192.53){\usebox{\plotpoint}}
\put(932.83,190.94){\usebox{\plotpoint}}
\put(953.53,189.34){\usebox{\plotpoint}}
\put(974.22,187.75){\usebox{\plotpoint}}
\put(994.92,186.16){\usebox{\plotpoint}}
\put(1015.65,185.57){\usebox{\plotpoint}}
\put(1036.34,183.97){\usebox{\plotpoint}}
\put(1057.04,182.38){\usebox{\plotpoint}}
\put(1077.77,181.79){\usebox{\plotpoint}}
\put(1098.46,180.20){\usebox{\plotpoint}}
\put(1119.20,179.60){\usebox{\plotpoint}}
\put(1139.89,178.01){\usebox{\plotpoint}}
\put(1160.62,177.41){\usebox{\plotpoint}}
\put(1181.36,176.82){\usebox{\plotpoint}}
\put(1202.05,175.23){\usebox{\plotpoint}}
\put(1222.78,174.63){\usebox{\plotpoint}}
\put(1243.51,174.00){\usebox{\plotpoint}}
\put(1264.23,173.00){\usebox{\plotpoint}}
\put(1284.95,172.00){\usebox{\plotpoint}}
\put(1305.68,171.26){\usebox{\plotpoint}}
\put(1326.42,171.00){\usebox{\plotpoint}}
\put(1347.14,170.07){\usebox{\plotpoint}}
\put(1367.87,169.47){\usebox{\plotpoint}}
\put(1388.61,169.00){\usebox{\plotpoint}}
\put(1409.34,168.28){\usebox{\plotpoint}}
\put(1430.07,167.69){\usebox{\plotpoint}}
\put(1439,167){\usebox{\plotpoint}}
\sbox{\plotpoint}{\rule[-0.400pt]{0.800pt}{0.800pt}}%
\multiput(249.40,833.50)(0.526,-4.417){7}{\rule{0.127pt}{6.143pt}}
\multiput(246.34,846.25)(7.000,-39.250){2}{\rule{0.800pt}{3.071pt}}
\multiput(256.41,787.01)(0.509,-3.056){19}{\rule{0.123pt}{4.815pt}}
\multiput(253.34,797.01)(13.000,-65.005){2}{\rule{0.800pt}{2.408pt}}
\multiput(269.41,715.84)(0.509,-2.436){19}{\rule{0.123pt}{3.892pt}}
\multiput(266.34,723.92)(13.000,-51.921){2}{\rule{0.800pt}{1.946pt}}
\multiput(282.41,658.40)(0.509,-2.022){19}{\rule{0.123pt}{3.277pt}}
\multiput(279.34,665.20)(13.000,-43.199){2}{\rule{0.800pt}{1.638pt}}
\multiput(295.41,610.70)(0.509,-1.650){19}{\rule{0.123pt}{2.723pt}}
\multiput(292.34,616.35)(13.000,-35.348){2}{\rule{0.800pt}{1.362pt}}
\multiput(308.41,571.48)(0.509,-1.360){19}{\rule{0.123pt}{2.292pt}}
\multiput(305.34,576.24)(13.000,-29.242){2}{\rule{0.800pt}{1.146pt}}
\multiput(321.41,538.51)(0.509,-1.195){19}{\rule{0.123pt}{2.046pt}}
\multiput(318.34,542.75)(13.000,-25.753){2}{\rule{0.800pt}{1.023pt}}
\multiput(334.41,509.53)(0.509,-1.029){19}{\rule{0.123pt}{1.800pt}}
\multiput(331.34,513.26)(13.000,-22.264){2}{\rule{0.800pt}{0.900pt}}
\multiput(347.41,484.55)(0.509,-0.864){19}{\rule{0.123pt}{1.554pt}}
\multiput(344.34,487.77)(13.000,-18.775){2}{\rule{0.800pt}{0.777pt}}
\multiput(360.41,463.06)(0.509,-0.781){19}{\rule{0.123pt}{1.431pt}}
\multiput(357.34,466.03)(13.000,-17.030){2}{\rule{0.800pt}{0.715pt}}
\multiput(373.41,443.57)(0.509,-0.698){19}{\rule{0.123pt}{1.308pt}}
\multiput(370.34,446.29)(13.000,-15.286){2}{\rule{0.800pt}{0.654pt}}
\multiput(386.41,426.08)(0.509,-0.616){19}{\rule{0.123pt}{1.185pt}}
\multiput(383.34,428.54)(13.000,-13.541){2}{\rule{0.800pt}{0.592pt}}
\multiput(399.41,410.59)(0.509,-0.533){19}{\rule{0.123pt}{1.062pt}}
\multiput(396.34,412.80)(13.000,-11.797){2}{\rule{0.800pt}{0.531pt}}
\multiput(411.00,399.08)(0.492,-0.509){19}{\rule{1.000pt}{0.123pt}}
\multiput(411.00,399.34)(10.924,-13.000){2}{\rule{0.500pt}{0.800pt}}
\multiput(424.00,386.08)(0.589,-0.512){15}{\rule{1.145pt}{0.123pt}}
\multiput(424.00,386.34)(10.623,-11.000){2}{\rule{0.573pt}{0.800pt}}
\multiput(437.00,375.08)(0.589,-0.512){15}{\rule{1.145pt}{0.123pt}}
\multiput(437.00,375.34)(10.623,-11.000){2}{\rule{0.573pt}{0.800pt}}
\multiput(450.00,364.08)(0.654,-0.514){13}{\rule{1.240pt}{0.124pt}}
\multiput(450.00,364.34)(10.426,-10.000){2}{\rule{0.620pt}{0.800pt}}
\multiput(463.00,354.08)(0.737,-0.516){11}{\rule{1.356pt}{0.124pt}}
\multiput(463.00,354.34)(10.186,-9.000){2}{\rule{0.678pt}{0.800pt}}
\multiput(476.00,345.08)(0.847,-0.520){9}{\rule{1.500pt}{0.125pt}}
\multiput(476.00,345.34)(9.887,-8.000){2}{\rule{0.750pt}{0.800pt}}
\multiput(489.00,337.08)(0.847,-0.520){9}{\rule{1.500pt}{0.125pt}}
\multiput(489.00,337.34)(9.887,-8.000){2}{\rule{0.750pt}{0.800pt}}
\multiput(502.00,329.08)(1.000,-0.526){7}{\rule{1.686pt}{0.127pt}}
\multiput(502.00,329.34)(9.501,-7.000){2}{\rule{0.843pt}{0.800pt}}
\multiput(515.00,322.08)(1.000,-0.526){7}{\rule{1.686pt}{0.127pt}}
\multiput(515.00,322.34)(9.501,-7.000){2}{\rule{0.843pt}{0.800pt}}
\multiput(528.00,315.07)(1.244,-0.536){5}{\rule{1.933pt}{0.129pt}}
\multiput(528.00,315.34)(8.987,-6.000){2}{\rule{0.967pt}{0.800pt}}
\multiput(541.00,309.07)(1.244,-0.536){5}{\rule{1.933pt}{0.129pt}}
\multiput(541.00,309.34)(8.987,-6.000){2}{\rule{0.967pt}{0.800pt}}
\multiput(554.00,303.06)(1.768,-0.560){3}{\rule{2.280pt}{0.135pt}}
\multiput(554.00,303.34)(8.268,-5.000){2}{\rule{1.140pt}{0.800pt}}
\multiput(567.00,298.06)(1.768,-0.560){3}{\rule{2.280pt}{0.135pt}}
\multiput(567.00,298.34)(8.268,-5.000){2}{\rule{1.140pt}{0.800pt}}
\multiput(580.00,293.06)(1.768,-0.560){3}{\rule{2.280pt}{0.135pt}}
\multiput(580.00,293.34)(8.268,-5.000){2}{\rule{1.140pt}{0.800pt}}
\multiput(593.00,288.06)(1.768,-0.560){3}{\rule{2.280pt}{0.135pt}}
\multiput(593.00,288.34)(8.268,-5.000){2}{\rule{1.140pt}{0.800pt}}
\put(606,281.34){\rule{2.800pt}{0.800pt}}
\multiput(606.00,283.34)(7.188,-4.000){2}{\rule{1.400pt}{0.800pt}}
\put(619,277.34){\rule{2.800pt}{0.800pt}}
\multiput(619.00,279.34)(7.188,-4.000){2}{\rule{1.400pt}{0.800pt}}
\put(632,273.34){\rule{2.800pt}{0.800pt}}
\multiput(632.00,275.34)(7.188,-4.000){2}{\rule{1.400pt}{0.800pt}}
\put(645,269.84){\rule{3.132pt}{0.800pt}}
\multiput(645.00,271.34)(6.500,-3.000){2}{\rule{1.566pt}{0.800pt}}
\put(658,266.34){\rule{2.800pt}{0.800pt}}
\multiput(658.00,268.34)(7.188,-4.000){2}{\rule{1.400pt}{0.800pt}}
\put(671,262.84){\rule{3.132pt}{0.800pt}}
\multiput(671.00,264.34)(6.500,-3.000){2}{\rule{1.566pt}{0.800pt}}
\put(684,259.84){\rule{3.132pt}{0.800pt}}
\multiput(684.00,261.34)(6.500,-3.000){2}{\rule{1.566pt}{0.800pt}}
\put(697,256.84){\rule{3.132pt}{0.800pt}}
\multiput(697.00,258.34)(6.500,-3.000){2}{\rule{1.566pt}{0.800pt}}
\put(710,253.84){\rule{3.132pt}{0.800pt}}
\multiput(710.00,255.34)(6.500,-3.000){2}{\rule{1.566pt}{0.800pt}}
\put(723,250.84){\rule{3.132pt}{0.800pt}}
\multiput(723.00,252.34)(6.500,-3.000){2}{\rule{1.566pt}{0.800pt}}
\put(736,248.34){\rule{3.132pt}{0.800pt}}
\multiput(736.00,249.34)(6.500,-2.000){2}{\rule{1.566pt}{0.800pt}}
\put(749,245.84){\rule{3.132pt}{0.800pt}}
\multiput(749.00,247.34)(6.500,-3.000){2}{\rule{1.566pt}{0.800pt}}
\put(762,243.34){\rule{3.132pt}{0.800pt}}
\multiput(762.00,244.34)(6.500,-2.000){2}{\rule{1.566pt}{0.800pt}}
\put(775,240.84){\rule{3.132pt}{0.800pt}}
\multiput(775.00,242.34)(6.500,-3.000){2}{\rule{1.566pt}{0.800pt}}
\put(788,238.34){\rule{3.373pt}{0.800pt}}
\multiput(788.00,239.34)(7.000,-2.000){2}{\rule{1.686pt}{0.800pt}}
\put(802,236.34){\rule{3.132pt}{0.800pt}}
\multiput(802.00,237.34)(6.500,-2.000){2}{\rule{1.566pt}{0.800pt}}
\put(815,234.34){\rule{3.132pt}{0.800pt}}
\multiput(815.00,235.34)(6.500,-2.000){2}{\rule{1.566pt}{0.800pt}}
\put(828,232.34){\rule{3.132pt}{0.800pt}}
\multiput(828.00,233.34)(6.500,-2.000){2}{\rule{1.566pt}{0.800pt}}
\put(841,230.34){\rule{3.132pt}{0.800pt}}
\multiput(841.00,231.34)(6.500,-2.000){2}{\rule{1.566pt}{0.800pt}}
\put(854,228.34){\rule{3.132pt}{0.800pt}}
\multiput(854.00,229.34)(6.500,-2.000){2}{\rule{1.566pt}{0.800pt}}
\put(867,226.84){\rule{3.132pt}{0.800pt}}
\multiput(867.00,227.34)(6.500,-1.000){2}{\rule{1.566pt}{0.800pt}}
\put(880,225.34){\rule{3.132pt}{0.800pt}}
\multiput(880.00,226.34)(6.500,-2.000){2}{\rule{1.566pt}{0.800pt}}
\put(893,223.34){\rule{3.132pt}{0.800pt}}
\multiput(893.00,224.34)(6.500,-2.000){2}{\rule{1.566pt}{0.800pt}}
\put(906,221.84){\rule{3.132pt}{0.800pt}}
\multiput(906.00,222.34)(6.500,-1.000){2}{\rule{1.566pt}{0.800pt}}
\put(919,220.34){\rule{3.132pt}{0.800pt}}
\multiput(919.00,221.34)(6.500,-2.000){2}{\rule{1.566pt}{0.800pt}}
\put(932,218.84){\rule{3.132pt}{0.800pt}}
\multiput(932.00,219.34)(6.500,-1.000){2}{\rule{1.566pt}{0.800pt}}
\put(945,217.34){\rule{3.132pt}{0.800pt}}
\multiput(945.00,218.34)(6.500,-2.000){2}{\rule{1.566pt}{0.800pt}}
\put(958,215.84){\rule{3.132pt}{0.800pt}}
\multiput(958.00,216.34)(6.500,-1.000){2}{\rule{1.566pt}{0.800pt}}
\put(971,214.34){\rule{3.132pt}{0.800pt}}
\multiput(971.00,215.34)(6.500,-2.000){2}{\rule{1.566pt}{0.800pt}}
\put(984,212.84){\rule{3.132pt}{0.800pt}}
\multiput(984.00,213.34)(6.500,-1.000){2}{\rule{1.566pt}{0.800pt}}
\put(997,211.84){\rule{3.132pt}{0.800pt}}
\multiput(997.00,212.34)(6.500,-1.000){2}{\rule{1.566pt}{0.800pt}}
\put(1010,210.84){\rule{3.132pt}{0.800pt}}
\multiput(1010.00,211.34)(6.500,-1.000){2}{\rule{1.566pt}{0.800pt}}
\put(1023,209.34){\rule{3.132pt}{0.800pt}}
\multiput(1023.00,210.34)(6.500,-2.000){2}{\rule{1.566pt}{0.800pt}}
\put(1036,207.84){\rule{3.132pt}{0.800pt}}
\multiput(1036.00,208.34)(6.500,-1.000){2}{\rule{1.566pt}{0.800pt}}
\put(1049,206.84){\rule{3.132pt}{0.800pt}}
\multiput(1049.00,207.34)(6.500,-1.000){2}{\rule{1.566pt}{0.800pt}}
\put(1062,205.84){\rule{3.132pt}{0.800pt}}
\multiput(1062.00,206.34)(6.500,-1.000){2}{\rule{1.566pt}{0.800pt}}
\put(1075,204.84){\rule{3.132pt}{0.800pt}}
\multiput(1075.00,205.34)(6.500,-1.000){2}{\rule{1.566pt}{0.800pt}}
\put(1088,203.84){\rule{3.132pt}{0.800pt}}
\multiput(1088.00,204.34)(6.500,-1.000){2}{\rule{1.566pt}{0.800pt}}
\put(1101,202.84){\rule{3.132pt}{0.800pt}}
\multiput(1101.00,203.34)(6.500,-1.000){2}{\rule{1.566pt}{0.800pt}}
\put(1114,201.84){\rule{3.132pt}{0.800pt}}
\multiput(1114.00,202.34)(6.500,-1.000){2}{\rule{1.566pt}{0.800pt}}
\put(1127,200.84){\rule{3.132pt}{0.800pt}}
\multiput(1127.00,201.34)(6.500,-1.000){2}{\rule{1.566pt}{0.800pt}}
\put(1140,199.84){\rule{3.132pt}{0.800pt}}
\multiput(1140.00,200.34)(6.500,-1.000){2}{\rule{1.566pt}{0.800pt}}
\put(1153,198.84){\rule{3.132pt}{0.800pt}}
\multiput(1153.00,199.34)(6.500,-1.000){2}{\rule{1.566pt}{0.800pt}}
\put(1166,197.84){\rule{3.132pt}{0.800pt}}
\multiput(1166.00,198.34)(6.500,-1.000){2}{\rule{1.566pt}{0.800pt}}
\put(1192,196.84){\rule{3.132pt}{0.800pt}}
\multiput(1192.00,197.34)(6.500,-1.000){2}{\rule{1.566pt}{0.800pt}}
\put(1205,195.84){\rule{3.132pt}{0.800pt}}
\multiput(1205.00,196.34)(6.500,-1.000){2}{\rule{1.566pt}{0.800pt}}
\put(1218,194.84){\rule{3.132pt}{0.800pt}}
\multiput(1218.00,195.34)(6.500,-1.000){2}{\rule{1.566pt}{0.800pt}}
\put(1231,193.84){\rule{3.132pt}{0.800pt}}
\multiput(1231.00,194.34)(6.500,-1.000){2}{\rule{1.566pt}{0.800pt}}
\put(1179.0,199.0){\rule[-0.400pt]{3.132pt}{0.800pt}}
\put(1257,192.84){\rule{3.132pt}{0.800pt}}
\multiput(1257.00,193.34)(6.500,-1.000){2}{\rule{1.566pt}{0.800pt}}
\put(1270,191.84){\rule{3.132pt}{0.800pt}}
\multiput(1270.00,192.34)(6.500,-1.000){2}{\rule{1.566pt}{0.800pt}}
\put(1283,190.84){\rule{3.132pt}{0.800pt}}
\multiput(1283.00,191.34)(6.500,-1.000){2}{\rule{1.566pt}{0.800pt}}
\put(1244.0,195.0){\rule[-0.400pt]{3.132pt}{0.800pt}}
\put(1309,189.84){\rule{3.132pt}{0.800pt}}
\multiput(1309.00,190.34)(6.500,-1.000){2}{\rule{1.566pt}{0.800pt}}
\put(1322,188.84){\rule{3.132pt}{0.800pt}}
\multiput(1322.00,189.34)(6.500,-1.000){2}{\rule{1.566pt}{0.800pt}}
\put(1296.0,192.0){\rule[-0.400pt]{3.132pt}{0.800pt}}
\put(1348,187.84){\rule{3.132pt}{0.800pt}}
\multiput(1348.00,188.34)(6.500,-1.000){2}{\rule{1.566pt}{0.800pt}}
\put(1335.0,190.0){\rule[-0.400pt]{3.132pt}{0.800pt}}
\put(1374,186.84){\rule{3.132pt}{0.800pt}}
\multiput(1374.00,187.34)(6.500,-1.000){2}{\rule{1.566pt}{0.800pt}}
\put(1387,185.84){\rule{3.132pt}{0.800pt}}
\multiput(1387.00,186.34)(6.500,-1.000){2}{\rule{1.566pt}{0.800pt}}
\put(1361.0,189.0){\rule[-0.400pt]{3.132pt}{0.800pt}}
\put(1413,184.84){\rule{3.132pt}{0.800pt}}
\multiput(1413.00,185.34)(6.500,-1.000){2}{\rule{1.566pt}{0.800pt}}
\put(1400.0,187.0){\rule[-0.400pt]{3.132pt}{0.800pt}}
\put(1426.0,186.0){\rule[-0.400pt]{3.132pt}{0.800pt}}
\sbox{\plotpoint}{\rule[-0.500pt]{1.000pt}{1.000pt}}%
\put(280.00,859.00){\usebox{\plotpoint}}
\multiput(281,852)(4.011,-20.364){3}{\usebox{\plotpoint}}
\multiput(294,786)(4.858,-20.179){3}{\usebox{\plotpoint}}
\multiput(307,732)(5.533,-20.004){2}{\usebox{\plotpoint}}
\multiput(320,685)(6.563,-19.690){2}{\usebox{\plotpoint}}
\multiput(333,646)(7.227,-19.457){2}{\usebox{\plotpoint}}
\put(351.06,599.32){\usebox{\plotpoint}}
\multiput(359,581)(9.282,-18.564){2}{\usebox{\plotpoint}}
\put(378.30,543.37){\usebox{\plotpoint}}
\put(388.52,525.31){\usebox{\plotpoint}}
\put(399.55,507.73){\usebox{\plotpoint}}
\multiput(411,491)(12.608,-16.487){2}{\usebox{\plotpoint}}
\put(437.49,458.44){\usebox{\plotpoint}}
\put(451.17,442.83){\usebox{\plotpoint}}
\put(465.95,428.27){\usebox{\plotpoint}}
\put(481.41,414.42){\usebox{\plotpoint}}
\put(497.57,401.41){\usebox{\plotpoint}}
\put(514.02,388.75){\usebox{\plotpoint}}
\put(531.50,377.58){\usebox{\plotpoint}}
\put(548.83,366.18){\usebox{\plotpoint}}
\put(566.93,356.04){\usebox{\plotpoint}}
\put(585.37,346.52){\usebox{\plotpoint}}
\put(604.22,337.82){\usebox{\plotpoint}}
\put(623.17,329.39){\usebox{\plotpoint}}
\put(642.55,321.94){\usebox{\plotpoint}}
\put(661.92,314.49){\usebox{\plotpoint}}
\put(681.54,307.76){\usebox{\plotpoint}}
\put(701.38,301.65){\usebox{\plotpoint}}
\put(721.21,295.55){\usebox{\plotpoint}}
\put(741.15,289.81){\usebox{\plotpoint}}
\put(761.14,284.27){\usebox{\plotpoint}}
\put(781.34,279.54){\usebox{\plotpoint}}
\put(801.62,275.08){\usebox{\plotpoint}}
\put(821.94,270.93){\usebox{\plotpoint}}
\put(842.25,266.71){\usebox{\plotpoint}}
\put(862.59,262.68){\usebox{\plotpoint}}
\put(883.06,259.29){\usebox{\plotpoint}}
\put(903.44,255.39){\usebox{\plotpoint}}
\put(923.95,252.24){\usebox{\plotpoint}}
\put(944.46,249.08){\usebox{\plotpoint}}
\put(964.98,245.93){\usebox{\plotpoint}}
\put(985.61,243.75){\usebox{\plotpoint}}
\put(1006.12,240.60){\usebox{\plotpoint}}
\put(1026.75,238.42){\usebox{\plotpoint}}
\put(1047.26,235.27){\usebox{\plotpoint}}
\put(1067.94,233.54){\usebox{\plotpoint}}
\put(1088.52,230.96){\usebox{\plotpoint}}
\put(1109.14,228.75){\usebox{\plotpoint}}
\put(1129.80,226.78){\usebox{\plotpoint}}
\put(1150.49,225.19){\usebox{\plotpoint}}
\put(1171.07,222.61){\usebox{\plotpoint}}
\put(1191.76,221.02){\usebox{\plotpoint}}
\put(1212.46,219.43){\usebox{\plotpoint}}
\put(1233.15,217.83){\usebox{\plotpoint}}
\put(1253.85,216.24){\usebox{\plotpoint}}
\put(1274.54,214.65){\usebox{\plotpoint}}
\put(1295.24,213.06){\usebox{\plotpoint}}
\put(1315.93,211.47){\usebox{\plotpoint}}
\put(1336.62,209.88){\usebox{\plotpoint}}
\put(1357.32,208.28){\usebox{\plotpoint}}
\put(1378.05,207.69){\usebox{\plotpoint}}
\put(1398.75,206.10){\usebox{\plotpoint}}
\put(1419.46,205.00){\usebox{\plotpoint}}
\put(1439,204){\usebox{\plotpoint}}
\sbox{\plotpoint}{\rule[-0.200pt]{0.400pt}{0.400pt}}%
\put(151.0,131.0){\rule[-0.200pt]{0.400pt}{175.375pt}}
\put(151.0,131.0){\rule[-0.200pt]{310.279pt}{0.400pt}}
\put(1439.0,131.0){\rule[-0.200pt]{0.400pt}{175.375pt}}
\put(151.0,859.0){\rule[-0.200pt]{310.279pt}{0.400pt}}
\end{picture}
}
    \caption{Sistema C}
  \end{subfigure}
  \caption{Algunas isotermas de cada sistema, tomando $n=R=C_c=C'_c=\theta =1$.}
\end{figure}
      
\section{Variable termométrica}
\begin{equation} \label{res}
\sqrt{\dfrac{\log{R'}}{\theta}}=a+b \log{R'}  
\end{equation}
Donde $a=-1.16$ y $b=0.675$
\subsection{}
\begin{center}
  $\theta =\dfrac {\log{R'}}{{(a+b \log{R'})}^2}$\\
  $\theta =\dfrac {\log{1000}}{{(a+b\log{1000})}^2}=4,0095$
\end{center}

\subsection{}
De la ecuación \eqref{res} se llega a lo sigueinte.\\
\begin{center}
  $R'=\exp_{10}{\Big(\dfrac{(1-2ab\theta)\pm \sqrt{1-4ab\theta}}{2b^2\theta}\Big)}$
\end{center}
Donde $\exp_{10}(x)$ es la función exponencial con base 10 ($10^x$).
\begin{figure}[H]
  \centering
  \scalebox{.9}{% GNUPLOT: LaTeX picture
\setlength{\unitlength}{0.240900pt}
\ifx\plotpoint\undefined\newsavebox{\plotpoint}\fi
\sbox{\plotpoint}{\rule[-0.200pt]{0.400pt}{0.400pt}}%
\begin{picture}(1500,900)(0,0)
\sbox{\plotpoint}{\rule[-0.200pt]{0.400pt}{0.400pt}}%
\put(130.0,82.0){\rule[-0.200pt]{4.818pt}{0.400pt}}
\put(110,82){\makebox(0,0)[r]{$1$}}
\put(1419.0,82.0){\rule[-0.200pt]{4.818pt}{0.400pt}}
\put(130.0,160.0){\rule[-0.200pt]{2.409pt}{0.400pt}}
\put(1429.0,160.0){\rule[-0.200pt]{2.409pt}{0.400pt}}
\put(130.0,206.0){\rule[-0.200pt]{2.409pt}{0.400pt}}
\put(1429.0,206.0){\rule[-0.200pt]{2.409pt}{0.400pt}}
\put(130.0,238.0){\rule[-0.200pt]{2.409pt}{0.400pt}}
\put(1429.0,238.0){\rule[-0.200pt]{2.409pt}{0.400pt}}
\put(130.0,263.0){\rule[-0.200pt]{2.409pt}{0.400pt}}
\put(1429.0,263.0){\rule[-0.200pt]{2.409pt}{0.400pt}}
\put(130.0,284.0){\rule[-0.200pt]{2.409pt}{0.400pt}}
\put(1429.0,284.0){\rule[-0.200pt]{2.409pt}{0.400pt}}
\put(130.0,301.0){\rule[-0.200pt]{2.409pt}{0.400pt}}
\put(1429.0,301.0){\rule[-0.200pt]{2.409pt}{0.400pt}}
\put(130.0,316.0){\rule[-0.200pt]{2.409pt}{0.400pt}}
\put(1429.0,316.0){\rule[-0.200pt]{2.409pt}{0.400pt}}
\put(130.0,329.0){\rule[-0.200pt]{2.409pt}{0.400pt}}
\put(1429.0,329.0){\rule[-0.200pt]{2.409pt}{0.400pt}}
\put(130.0,341.0){\rule[-0.200pt]{4.818pt}{0.400pt}}
\put(110,341){\makebox(0,0)[r]{$10$}}
\put(1419.0,341.0){\rule[-0.200pt]{4.818pt}{0.400pt}}
\put(130.0,419.0){\rule[-0.200pt]{2.409pt}{0.400pt}}
\put(1429.0,419.0){\rule[-0.200pt]{2.409pt}{0.400pt}}
\put(130.0,465.0){\rule[-0.200pt]{2.409pt}{0.400pt}}
\put(1429.0,465.0){\rule[-0.200pt]{2.409pt}{0.400pt}}
\put(130.0,497.0){\rule[-0.200pt]{2.409pt}{0.400pt}}
\put(1429.0,497.0){\rule[-0.200pt]{2.409pt}{0.400pt}}
\put(130.0,522.0){\rule[-0.200pt]{2.409pt}{0.400pt}}
\put(1429.0,522.0){\rule[-0.200pt]{2.409pt}{0.400pt}}
\put(130.0,543.0){\rule[-0.200pt]{2.409pt}{0.400pt}}
\put(1429.0,543.0){\rule[-0.200pt]{2.409pt}{0.400pt}}
\put(130.0,560.0){\rule[-0.200pt]{2.409pt}{0.400pt}}
\put(1429.0,560.0){\rule[-0.200pt]{2.409pt}{0.400pt}}
\put(130.0,575.0){\rule[-0.200pt]{2.409pt}{0.400pt}}
\put(1429.0,575.0){\rule[-0.200pt]{2.409pt}{0.400pt}}
\put(130.0,588.0){\rule[-0.200pt]{2.409pt}{0.400pt}}
\put(1429.0,588.0){\rule[-0.200pt]{2.409pt}{0.400pt}}
\put(130.0,600.0){\rule[-0.200pt]{4.818pt}{0.400pt}}
\put(110,600){\makebox(0,0)[r]{$100$}}
\put(1419.0,600.0){\rule[-0.200pt]{4.818pt}{0.400pt}}
\put(130.0,678.0){\rule[-0.200pt]{2.409pt}{0.400pt}}
\put(1429.0,678.0){\rule[-0.200pt]{2.409pt}{0.400pt}}
\put(130.0,724.0){\rule[-0.200pt]{2.409pt}{0.400pt}}
\put(1429.0,724.0){\rule[-0.200pt]{2.409pt}{0.400pt}}
\put(130.0,756.0){\rule[-0.200pt]{2.409pt}{0.400pt}}
\put(1429.0,756.0){\rule[-0.200pt]{2.409pt}{0.400pt}}
\put(130.0,781.0){\rule[-0.200pt]{2.409pt}{0.400pt}}
\put(1429.0,781.0){\rule[-0.200pt]{2.409pt}{0.400pt}}
\put(130.0,802.0){\rule[-0.200pt]{2.409pt}{0.400pt}}
\put(1429.0,802.0){\rule[-0.200pt]{2.409pt}{0.400pt}}
\put(130.0,819.0){\rule[-0.200pt]{2.409pt}{0.400pt}}
\put(1429.0,819.0){\rule[-0.200pt]{2.409pt}{0.400pt}}
\put(130.0,834.0){\rule[-0.200pt]{2.409pt}{0.400pt}}
\put(1429.0,834.0){\rule[-0.200pt]{2.409pt}{0.400pt}}
\put(130.0,847.0){\rule[-0.200pt]{2.409pt}{0.400pt}}
\put(1429.0,847.0){\rule[-0.200pt]{2.409pt}{0.400pt}}
\put(130.0,859.0){\rule[-0.200pt]{4.818pt}{0.400pt}}
\put(110,859){\makebox(0,0)[r]{$1000$}}
\put(1419.0,859.0){\rule[-0.200pt]{4.818pt}{0.400pt}}
\put(130.0,82.0){\rule[-0.200pt]{0.400pt}{4.818pt}}
\put(130,41){\makebox(0,0){$0.01$}}
\put(130.0,839.0){\rule[-0.200pt]{0.400pt}{4.818pt}}
\put(229.0,82.0){\rule[-0.200pt]{0.400pt}{2.409pt}}
\put(229.0,849.0){\rule[-0.200pt]{0.400pt}{2.409pt}}
\put(286.0,82.0){\rule[-0.200pt]{0.400pt}{2.409pt}}
\put(286.0,849.0){\rule[-0.200pt]{0.400pt}{2.409pt}}
\put(327.0,82.0){\rule[-0.200pt]{0.400pt}{2.409pt}}
\put(327.0,849.0){\rule[-0.200pt]{0.400pt}{2.409pt}}
\put(359.0,82.0){\rule[-0.200pt]{0.400pt}{2.409pt}}
\put(359.0,849.0){\rule[-0.200pt]{0.400pt}{2.409pt}}
\put(385.0,82.0){\rule[-0.200pt]{0.400pt}{2.409pt}}
\put(385.0,849.0){\rule[-0.200pt]{0.400pt}{2.409pt}}
\put(407.0,82.0){\rule[-0.200pt]{0.400pt}{2.409pt}}
\put(407.0,849.0){\rule[-0.200pt]{0.400pt}{2.409pt}}
\put(426.0,82.0){\rule[-0.200pt]{0.400pt}{2.409pt}}
\put(426.0,849.0){\rule[-0.200pt]{0.400pt}{2.409pt}}
\put(442.0,82.0){\rule[-0.200pt]{0.400pt}{2.409pt}}
\put(442.0,849.0){\rule[-0.200pt]{0.400pt}{2.409pt}}
\put(457.0,82.0){\rule[-0.200pt]{0.400pt}{4.818pt}}
\put(457,41){\makebox(0,0){$0.1$}}
\put(457.0,839.0){\rule[-0.200pt]{0.400pt}{4.818pt}}
\put(556.0,82.0){\rule[-0.200pt]{0.400pt}{2.409pt}}
\put(556.0,849.0){\rule[-0.200pt]{0.400pt}{2.409pt}}
\put(613.0,82.0){\rule[-0.200pt]{0.400pt}{2.409pt}}
\put(613.0,849.0){\rule[-0.200pt]{0.400pt}{2.409pt}}
\put(654.0,82.0){\rule[-0.200pt]{0.400pt}{2.409pt}}
\put(654.0,849.0){\rule[-0.200pt]{0.400pt}{2.409pt}}
\put(686.0,82.0){\rule[-0.200pt]{0.400pt}{2.409pt}}
\put(686.0,849.0){\rule[-0.200pt]{0.400pt}{2.409pt}}
\put(712.0,82.0){\rule[-0.200pt]{0.400pt}{2.409pt}}
\put(712.0,849.0){\rule[-0.200pt]{0.400pt}{2.409pt}}
\put(734.0,82.0){\rule[-0.200pt]{0.400pt}{2.409pt}}
\put(734.0,849.0){\rule[-0.200pt]{0.400pt}{2.409pt}}
\put(753.0,82.0){\rule[-0.200pt]{0.400pt}{2.409pt}}
\put(753.0,849.0){\rule[-0.200pt]{0.400pt}{2.409pt}}
\put(770.0,82.0){\rule[-0.200pt]{0.400pt}{2.409pt}}
\put(770.0,849.0){\rule[-0.200pt]{0.400pt}{2.409pt}}
\put(785.0,82.0){\rule[-0.200pt]{0.400pt}{4.818pt}}
\put(785,41){\makebox(0,0){$1$}}
\put(785.0,839.0){\rule[-0.200pt]{0.400pt}{4.818pt}}
\put(883.0,82.0){\rule[-0.200pt]{0.400pt}{2.409pt}}
\put(883.0,849.0){\rule[-0.200pt]{0.400pt}{2.409pt}}
\put(941.0,82.0){\rule[-0.200pt]{0.400pt}{2.409pt}}
\put(941.0,849.0){\rule[-0.200pt]{0.400pt}{2.409pt}}
\put(982.0,82.0){\rule[-0.200pt]{0.400pt}{2.409pt}}
\put(982.0,849.0){\rule[-0.200pt]{0.400pt}{2.409pt}}
\put(1013.0,82.0){\rule[-0.200pt]{0.400pt}{2.409pt}}
\put(1013.0,849.0){\rule[-0.200pt]{0.400pt}{2.409pt}}
\put(1039.0,82.0){\rule[-0.200pt]{0.400pt}{2.409pt}}
\put(1039.0,849.0){\rule[-0.200pt]{0.400pt}{2.409pt}}
\put(1061.0,82.0){\rule[-0.200pt]{0.400pt}{2.409pt}}
\put(1061.0,849.0){\rule[-0.200pt]{0.400pt}{2.409pt}}
\put(1080.0,82.0){\rule[-0.200pt]{0.400pt}{2.409pt}}
\put(1080.0,849.0){\rule[-0.200pt]{0.400pt}{2.409pt}}
\put(1097.0,82.0){\rule[-0.200pt]{0.400pt}{2.409pt}}
\put(1097.0,849.0){\rule[-0.200pt]{0.400pt}{2.409pt}}
\put(1112.0,82.0){\rule[-0.200pt]{0.400pt}{4.818pt}}
\put(1112,41){\makebox(0,0){$10$}}
\put(1112.0,839.0){\rule[-0.200pt]{0.400pt}{4.818pt}}
\put(1210.0,82.0){\rule[-0.200pt]{0.400pt}{2.409pt}}
\put(1210.0,849.0){\rule[-0.200pt]{0.400pt}{2.409pt}}
\put(1268.0,82.0){\rule[-0.200pt]{0.400pt}{2.409pt}}
\put(1268.0,849.0){\rule[-0.200pt]{0.400pt}{2.409pt}}
\put(1309.0,82.0){\rule[-0.200pt]{0.400pt}{2.409pt}}
\put(1309.0,849.0){\rule[-0.200pt]{0.400pt}{2.409pt}}
\put(1340.0,82.0){\rule[-0.200pt]{0.400pt}{2.409pt}}
\put(1340.0,849.0){\rule[-0.200pt]{0.400pt}{2.409pt}}
\put(1366.0,82.0){\rule[-0.200pt]{0.400pt}{2.409pt}}
\put(1366.0,849.0){\rule[-0.200pt]{0.400pt}{2.409pt}}
\put(1388.0,82.0){\rule[-0.200pt]{0.400pt}{2.409pt}}
\put(1388.0,849.0){\rule[-0.200pt]{0.400pt}{2.409pt}}
\put(1407.0,82.0){\rule[-0.200pt]{0.400pt}{2.409pt}}
\put(1407.0,849.0){\rule[-0.200pt]{0.400pt}{2.409pt}}
\put(1424.0,82.0){\rule[-0.200pt]{0.400pt}{2.409pt}}
\put(1424.0,849.0){\rule[-0.200pt]{0.400pt}{2.409pt}}
\put(1439.0,82.0){\rule[-0.200pt]{0.400pt}{4.818pt}}
\put(1439,41){\makebox(0,0){$100$}}
\put(1439.0,839.0){\rule[-0.200pt]{0.400pt}{4.818pt}}
\put(130.0,82.0){\rule[-0.200pt]{0.400pt}{187.179pt}}
\put(130.0,82.0){\rule[-0.200pt]{315.338pt}{0.400pt}}
\put(1439.0,82.0){\rule[-0.200pt]{0.400pt}{187.179pt}}
\put(130.0,859.0){\rule[-0.200pt]{315.338pt}{0.400pt}}
\put(450,163){\makebox(0,0)[r]{Ra\'iz positiva}}
\multiput(982.59,855.98)(0.485,-0.798){11}{\rule{0.117pt}{0.729pt}}
\multiput(981.17,857.49)(7.000,-9.488){2}{\rule{0.400pt}{0.364pt}}
\multiput(989.58,845.45)(0.494,-0.644){25}{\rule{0.119pt}{0.614pt}}
\multiput(988.17,846.73)(14.000,-16.725){2}{\rule{0.400pt}{0.307pt}}
\multiput(1003.58,827.41)(0.493,-0.655){23}{\rule{0.119pt}{0.623pt}}
\multiput(1002.17,828.71)(13.000,-15.707){2}{\rule{0.400pt}{0.312pt}}
\multiput(1016.58,810.54)(0.493,-0.616){23}{\rule{0.119pt}{0.592pt}}
\multiput(1015.17,811.77)(13.000,-14.771){2}{\rule{0.400pt}{0.296pt}}
\multiput(1029.58,794.67)(0.493,-0.576){23}{\rule{0.119pt}{0.562pt}}
\multiput(1028.17,795.83)(13.000,-13.834){2}{\rule{0.400pt}{0.281pt}}
\multiput(1042.00,780.92)(0.497,-0.494){25}{\rule{0.500pt}{0.119pt}}
\multiput(1042.00,781.17)(12.962,-14.000){2}{\rule{0.250pt}{0.400pt}}
\multiput(1056.00,766.92)(0.497,-0.493){23}{\rule{0.500pt}{0.119pt}}
\multiput(1056.00,767.17)(11.962,-13.000){2}{\rule{0.250pt}{0.400pt}}
\multiput(1069.00,753.92)(0.497,-0.493){23}{\rule{0.500pt}{0.119pt}}
\multiput(1069.00,754.17)(11.962,-13.000){2}{\rule{0.250pt}{0.400pt}}
\multiput(1082.00,740.92)(0.590,-0.492){19}{\rule{0.573pt}{0.118pt}}
\multiput(1082.00,741.17)(11.811,-11.000){2}{\rule{0.286pt}{0.400pt}}
\multiput(1095.00,729.92)(0.590,-0.492){19}{\rule{0.573pt}{0.118pt}}
\multiput(1095.00,730.17)(11.811,-11.000){2}{\rule{0.286pt}{0.400pt}}
\multiput(1108.00,718.92)(0.637,-0.492){19}{\rule{0.609pt}{0.118pt}}
\multiput(1108.00,719.17)(12.736,-11.000){2}{\rule{0.305pt}{0.400pt}}
\multiput(1122.00,707.93)(0.728,-0.489){15}{\rule{0.678pt}{0.118pt}}
\multiput(1122.00,708.17)(11.593,-9.000){2}{\rule{0.339pt}{0.400pt}}
\multiput(1135.00,698.93)(0.728,-0.489){15}{\rule{0.678pt}{0.118pt}}
\multiput(1135.00,699.17)(11.593,-9.000){2}{\rule{0.339pt}{0.400pt}}
\multiput(1148.00,689.93)(0.728,-0.489){15}{\rule{0.678pt}{0.118pt}}
\multiput(1148.00,690.17)(11.593,-9.000){2}{\rule{0.339pt}{0.400pt}}
\multiput(1161.00,680.93)(0.890,-0.488){13}{\rule{0.800pt}{0.117pt}}
\multiput(1161.00,681.17)(12.340,-8.000){2}{\rule{0.400pt}{0.400pt}}
\multiput(1175.00,672.93)(0.950,-0.485){11}{\rule{0.843pt}{0.117pt}}
\multiput(1175.00,673.17)(11.251,-7.000){2}{\rule{0.421pt}{0.400pt}}
\multiput(1188.00,665.93)(0.824,-0.488){13}{\rule{0.750pt}{0.117pt}}
\multiput(1188.00,666.17)(11.443,-8.000){2}{\rule{0.375pt}{0.400pt}}
\multiput(1201.00,657.93)(1.123,-0.482){9}{\rule{0.967pt}{0.116pt}}
\multiput(1201.00,658.17)(10.994,-6.000){2}{\rule{0.483pt}{0.400pt}}
\multiput(1214.00,651.93)(0.950,-0.485){11}{\rule{0.843pt}{0.117pt}}
\multiput(1214.00,652.17)(11.251,-7.000){2}{\rule{0.421pt}{0.400pt}}
\multiput(1227.00,644.93)(1.214,-0.482){9}{\rule{1.033pt}{0.116pt}}
\multiput(1227.00,645.17)(11.855,-6.000){2}{\rule{0.517pt}{0.400pt}}
\multiput(1241.00,638.93)(1.378,-0.477){7}{\rule{1.140pt}{0.115pt}}
\multiput(1241.00,639.17)(10.634,-5.000){2}{\rule{0.570pt}{0.400pt}}
\multiput(1254.00,633.93)(1.123,-0.482){9}{\rule{0.967pt}{0.116pt}}
\multiput(1254.00,634.17)(10.994,-6.000){2}{\rule{0.483pt}{0.400pt}}
\multiput(1267.00,627.93)(1.378,-0.477){7}{\rule{1.140pt}{0.115pt}}
\multiput(1267.00,628.17)(10.634,-5.000){2}{\rule{0.570pt}{0.400pt}}
\multiput(1280.00,622.93)(1.489,-0.477){7}{\rule{1.220pt}{0.115pt}}
\multiput(1280.00,623.17)(11.468,-5.000){2}{\rule{0.610pt}{0.400pt}}
\multiput(1294.00,617.94)(1.797,-0.468){5}{\rule{1.400pt}{0.113pt}}
\multiput(1294.00,618.17)(10.094,-4.000){2}{\rule{0.700pt}{0.400pt}}
\multiput(1307.00,613.93)(1.378,-0.477){7}{\rule{1.140pt}{0.115pt}}
\multiput(1307.00,614.17)(10.634,-5.000){2}{\rule{0.570pt}{0.400pt}}
\multiput(1320.00,608.94)(1.797,-0.468){5}{\rule{1.400pt}{0.113pt}}
\multiput(1320.00,609.17)(10.094,-4.000){2}{\rule{0.700pt}{0.400pt}}
\multiput(1333.00,604.94)(1.797,-0.468){5}{\rule{1.400pt}{0.113pt}}
\multiput(1333.00,605.17)(10.094,-4.000){2}{\rule{0.700pt}{0.400pt}}
\multiput(1346.00,600.95)(2.918,-0.447){3}{\rule{1.967pt}{0.108pt}}
\multiput(1346.00,601.17)(9.918,-3.000){2}{\rule{0.983pt}{0.400pt}}
\multiput(1360.00,597.94)(1.797,-0.468){5}{\rule{1.400pt}{0.113pt}}
\multiput(1360.00,598.17)(10.094,-4.000){2}{\rule{0.700pt}{0.400pt}}
\multiput(1373.00,593.95)(2.695,-0.447){3}{\rule{1.833pt}{0.108pt}}
\multiput(1373.00,594.17)(9.195,-3.000){2}{\rule{0.917pt}{0.400pt}}
\multiput(1386.00,590.95)(2.695,-0.447){3}{\rule{1.833pt}{0.108pt}}
\multiput(1386.00,591.17)(9.195,-3.000){2}{\rule{0.917pt}{0.400pt}}
\multiput(1399.00,587.95)(2.918,-0.447){3}{\rule{1.967pt}{0.108pt}}
\multiput(1399.00,588.17)(9.918,-3.000){2}{\rule{0.983pt}{0.400pt}}
\multiput(1413.00,584.95)(2.695,-0.447){3}{\rule{1.833pt}{0.108pt}}
\multiput(1413.00,585.17)(9.195,-3.000){2}{\rule{0.917pt}{0.400pt}}
\multiput(1426.00,581.95)(2.695,-0.447){3}{\rule{1.833pt}{0.108pt}}
\multiput(1426.00,582.17)(9.195,-3.000){2}{\rule{0.917pt}{0.400pt}}
\put(470.0,163.0){\rule[-0.200pt]{24.090pt}{0.400pt}}
\put(450,122){\makebox(0,0)[r]{Ra\'iz negativa}}
\multiput(470,122)(20.756,0.000){5}{\usebox{\plotpoint}}
\put(570,122){\usebox{\plotpoint}}
\put(130,85){\usebox{\plotpoint}}
\put(130.00,85.00){\usebox{\plotpoint}}
\put(150.72,86.00){\usebox{\plotpoint}}
\put(171.44,87.00){\usebox{\plotpoint}}
\put(192.19,87.00){\usebox{\plotpoint}}
\put(212.91,88.00){\usebox{\plotpoint}}
\put(233.63,88.82){\usebox{\plotpoint}}
\put(254.33,90.41){\usebox{\plotpoint}}
\put(275.06,91.00){\usebox{\plotpoint}}
\put(295.76,92.52){\usebox{\plotpoint}}
\put(316.46,94.11){\usebox{\plotpoint}}
\put(337.15,95.65){\usebox{\plotpoint}}
\put(357.74,98.21){\usebox{\plotpoint}}
\put(378.34,100.59){\usebox{\plotpoint}}
\put(398.97,102.71){\usebox{\plotpoint}}
\put(419.50,105.77){\usebox{\plotpoint}}
\put(440.02,108.93){\usebox{\plotpoint}}
\put(460.39,112.87){\usebox{\plotpoint}}
\put(480.80,116.57){\usebox{\plotpoint}}
\put(501.02,121.24){\usebox{\plotpoint}}
\put(521.14,126.32){\usebox{\plotpoint}}
\put(541.26,131.39){\usebox{\plotpoint}}
\put(561.09,137.49){\usebox{\plotpoint}}
\put(580.99,143.38){\usebox{\plotpoint}}
\put(600.54,150.32){\usebox{\plotpoint}}
\put(620.04,157.40){\usebox{\plotpoint}}
\put(639.30,165.13){\usebox{\plotpoint}}
\put(658.57,172.83){\usebox{\plotpoint}}
\put(677.43,181.50){\usebox{\plotpoint}}
\put(696.41,189.89){\usebox{\plotpoint}}
\put(715.29,198.52){\usebox{\plotpoint}}
\put(733.85,207.77){\usebox{\plotpoint}}
\put(752.55,216.77){\usebox{\plotpoint}}
\put(771.02,226.24){\usebox{\plotpoint}}
\put(789.29,236.08){\usebox{\plotpoint}}
\put(807.62,245.81){\usebox{\plotpoint}}
\put(826.06,255.34){\usebox{\plotpoint}}
\put(844.33,265.18){\usebox{\plotpoint}}
\put(862.61,275.02){\usebox{\plotpoint}}
\put(880.73,285.13){\usebox{\plotpoint}}
\put(898.96,295.06){\usebox{\plotpoint}}
\put(917.46,304.44){\usebox{\plotpoint}}
\put(936.11,313.55){\usebox{\plotpoint}}
\put(954.40,323.37){\usebox{\plotpoint}}
\put(972.97,332.60){\usebox{\plotpoint}}
\put(991.44,342.05){\usebox{\plotpoint}}
\put(1010.43,350.43){\usebox{\plotpoint}}
\put(1029.27,359.13){\usebox{\plotpoint}}
\put(1048.19,367.65){\usebox{\plotpoint}}
\put(1067.13,376.14){\usebox{\plotpoint}}
\put(1086.45,383.71){\usebox{\plotpoint}}
\put(1105.53,391.86){\usebox{\plotpoint}}
\put(1124.96,399.14){\usebox{\plotpoint}}
\put(1144.55,405.94){\usebox{\plotpoint}}
\put(1164.03,413.08){\usebox{\plotpoint}}
\put(1183.71,419.68){\usebox{\plotpoint}}
\put(1203.55,425.78){\usebox{\plotpoint}}
\put(1223.38,431.89){\usebox{\plotpoint}}
\put(1243.31,437.71){\usebox{\plotpoint}}
\put(1263.32,443.15){\usebox{\plotpoint}}
\put(1283.50,448.00){\usebox{\plotpoint}}
\put(1303.58,453.21){\usebox{\plotpoint}}
\put(1323.81,457.88){\usebox{\plotpoint}}
\put(1344.03,462.55){\usebox{\plotpoint}}
\put(1364.47,466.03){\usebox{\plotpoint}}
\put(1384.87,469.83){\usebox{\plotpoint}}
\put(1405.20,473.89){\usebox{\plotpoint}}
\put(1425.73,476.96){\usebox{\plotpoint}}
\put(1439,480){\usebox{\plotpoint}}
\put(130.0,82.0){\rule[-0.200pt]{0.400pt}{187.179pt}}
\put(130.0,82.0){\rule[-0.200pt]{315.338pt}{0.400pt}}
\put(1439.0,82.0){\rule[-0.200pt]{0.400pt}{187.179pt}}
\put(130.0,859.0){\rule[-0.200pt]{315.338pt}{0.400pt}}
\end{picture}
}
  \caption{Gráfico Log-Log de $R'$ en función de $\theta$, para $1000 \Omega \leq R' \leq 30000 \Omega$}
\end{figure}
\section{}
La temperatura promedio de la atmósfera se aproxima mediante la relación\\
\begin{center}
  $T_{atm}=288.15-6.5z$
\end{center}
Donde $T_{atm}$ es la temperatura de la atmósfera medida en K, y Z es la altura en Km.
La temperatura de un avión que viaja a 12,000m, viene dada por,
\begin{center}
  $T_{atm}=288.15-6.5(12)=210.15$
\end{center}
\end{document}

